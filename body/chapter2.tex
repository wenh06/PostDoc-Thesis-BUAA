%\mbox{}\newpage
\chapter{\hspace{-1mm}\bf 联邦学习中的优化算法}
\label{chap2}
\addcontentsline{toe}{chapter}{{{\bf Chapter 2\ \
Optimization Algorithms in Federated Learning }}\numberline\,}
\markboth{第\,2\,章\ \
待写}{北京航空航天大学博士后研究工作报告}

%%%%%%%%%%%%%%%%%%%%%%%%%%%%%%%%%%%%%%%%%%%%%%%%%%%%%%%%%%%

\section{联邦学习中的优化算法}
\addcontentsline{toe}{section}{{2.1\ \ To write}\numberline\,}

优化算法自从联邦学习这一概念诞生起,便一直是其中心问题。联邦学习开创性的文章\cite{mcmahan2017fed_avg}中最重要也是最为人所熟知的便是优化算法FedAvg的提出。

\section{联邦学习中的原始对偶算法}
\addcontentsline{toe}{section}{{2.2\ \ Primal-Dual Algorithms in Federated Learning}\numberline\,}

待写。。。。

\section{联邦学习中的算子分裂算法}
\addcontentsline{toe}{section}{{2.3\ \ Operator Splitting Algorithms in Federated Learning}\numberline\,}

待写。。。。
