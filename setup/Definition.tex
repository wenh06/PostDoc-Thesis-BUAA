%===================== 重定义字体、字号命令 =============================%
\newCJKfontfamily\simfang{simfang.ttf}[Extension = .ttf, Path=fonts/]
\newCJKfontfamily\simhei{simhei.ttf}[Extension = .ttf, Path=fonts/]
\newCJKfontfamily\simkai{simkai.ttf}[Extension = .ttf, Path=fonts/]
\newCJKfontfamily\simsun{simsun.ttc}[Extension = .ttc, Path=fonts/]

\newcommand{\song}{\simsun}     % 宋体   (Windows自带simsun.ttf)
\newcommand{\fs}{\simfang}      % 仿宋体 (Windows自带simfs.ttf)
\newcommand{\kai}{\simkai}      % 楷体   (Windows自带simkai.ttf)
\newcommand{\hei}{\simhei}      % 黑体   (Windows自带simhei.ttf)
% \newcommand{\li}{\CJKfamily{li}}        % 隶书   (Windows自带simli.ttf)
% \newcommand{\you}{\CJKfamily{you}}      % 幼圆   (Windows自带simyou.ttf)
\newcommand{\chuhao}{\fontsize{42pt}{\baselineskip}\selectfont}           % 字号设置
\newcommand{\xiaochuhao}{\fontsize{36pt}{\baselineskip}\selectfont}       % 字号设置
\newcommand{\yihao}{\fontsize{28pt}{\baselineskip}\selectfont}            % 字号设置
\newcommand{\erhao}{\fontsize{22pt}{\baselineskip}\selectfont}            % 字号设置
\newcommand{\xiaoerhao}{\fontsize{18pt}{\baselineskip}\selectfont}        % 字号设置
\newcommand{\sanhao}{\fontsize{15.75pt}{\baselineskip}\selectfont}        % 字号设置
\newcommand{\sihao}{\fontsize{14pt}{\baselineskip}\selectfont}            % 字号设置
%\newcommand{\xiaosihao}{\fontsize{12pt}{20pt}\selectfont}                % 字号设置
\newcommand{\xiaosihao}{\fontsize{12pt}{14pt}\selectfont}                 % 字号设置
\newcommand{\wuhao}{\fontsize{10.5pt}{12.6pt}\selectfont}                 % 字号设置
\newcommand{\xiaowuhao}{\fontsize{9pt}{11pt}{\baselineskip}\selectfont}   % 字号设置
\newcommand{\liuhao}{\fontsize{7.875pt}{\baselineskip}\selectfont}        % 字号设置
\newcommand{\qihao}{\fontsize{5.25pt}{\baselineskip}\selectfont}          % 字号设置
%\iffalse

\makeatletter

%%%%%%%%%%%%%%%%%%%%%%%%%%%%%%%%%%%%%将目录中的第一章改成第1 章%%%%%%%%%%%%%%%%%%%%%%%%%%%%%%%%%%%%

\renewcommand{\chaptername}{第~\@arabic\c@chapter~章}
\renewcommand{\@makechapterhead}[1]{%
   \vspace*{-\baselineskip}%
   {\normalfont \flushleft\Large\bfseries%
   \chaptername \quad #1 \par\nobreak%
   \vspace{1.5\baselineskip}
   }}
\renewcommand{\@makeschapterhead}[1]{%
   \vspace*{-\baselineskip}%
   {\normalfont \flushleft\Large\bfseries #1 \par\nobreak%
   \vspace{1.5\baselineskip}
   }}

\def\@chapter[#1]#2{\ifnum \c@secnumdepth >\m@ne
                        \if@mainmatter
                          \refstepcounter{chapter}%
                          \typeout{第~\thechapter~章}%
                          \addcontentsline{toc}{chapter}%
                                    {\protect\numberline{}%
                                     第~\expandafter\noexpand\thechapter~ 章\hspace{0.8em}#1}%
                        \else
                          \addcontentsline{toc}{chapter}{#1}%
                        \fi
                     \else
                       \addcontentsline{toc}{chapter}{#1}%
                     \fi
                     \chaptermark{#1}%
                     \addtocontents{lof}{\protect\addvspace{10\p@}}%
                     \addtocontents{lot}{\protect\addvspace{10\p@}}%
                     \if@twocolumn
                       \@topnewpage[\@makechapterhead{#2}]%
                     \else
                       \@makechapterhead{#2}%
                       \@afterheading
                     \fi}

\makeatother

%画两条页眉线
%\pagestyle{plain}
 \pagestyle{fancy}
\newcommand{\makeheadrule}{%
    \makebox[0pt][l]{\rule[.7\baselineskip]{\headwidth}{0.5pt}}%
    \rule[.6\baselineskip]{\headwidth}{0.5pt}\vskip-.8\baselineskip}

\makeatletter
\renewcommand{\headrule}{%
    {\if@fancyplain\let\headrulewidth\plainheadrulewidth\fi
     \makeheadrule}}

%定义普通页眉
\fancyhf{}
\renewcommand{\chaptermark}[1]{\markboth{第\thechapter 章\ #1}{}}
\fancyhead[CO]{\song \leftmark}
\fancyhead[CE]{\song 北京航空航天大学博士后出站报告}
\fancyfoot[CE, CO]{\thepage}

%定义章首页的页眉和页脚.
\fancypagestyle{plain}{%
\fancyhead{} % clear all header fields
\fancyhead[CO]{\song \leftmark}
\fancyhead[CE]{\song 北京航空航天大学博士后研究工作报告}
\fancyfoot[CE, CO]{\thepage}}

%定义标题格式
\titleformat{\chapter}
   % {\normalfont\bfseries\huge\filcenter\CJKfamily{hei}}
    {\normalfont\LARGE\filcenter\rm}
    %{\huge{\chaptertitlename}}
    %{第~\CJKnumber{\thechapter}~章}
    {\bf 第\,{\thechapter}\,章}
    {18pt}{\rm}
\titlespacing{\chapter}{0pt}{-3ex  plus .1ex minus .2ex}{1.5ex plus .1ex minus .2ex}

\titleformat{\section}[hang]{\Large\bf}% add \bfseries if you want to use bold fonts
    {\Large \ \thesection}{1em}{}{}
\titlespacing{\section}%
    {0pt}{1.5ex plus .1ex minus .2ex}{\wordsep}%{1ex plus .1ex minus .2ex}

\titleformat{\subsection}[hang]{\large\bf}
    {\large\ \thesubsection}{1em}{}{}
\titlespacing{\subsection}%
    {0pt}{1.5ex plus .1ex minus .2ex}{\wordsep}

\raggedbottom
\parskip 0.1cm
\parindent 0.8cm

\numberwithin{equation}{section}
\newtheorem{theorem}{{\heiti 定理}}[section]
\newtheorem{definition}{{\heiti 定义}}[section]
\newtheorem{lemma}{{\heiti 引理}}[section]
\newtheorem{corollary}{{\heiti 推论}}[section]
\newtheorem{property}{{\heiti 性质}}[section]
\newtheorem{prop}{{\heiti 命题}}[section]
\newtheorem{assu}{{\heiti 假设}}[section]
\newtheorem{Proof}{{\heiti 证明}}[section]
\newtheorem{rem}{{\heiti 注记}}[section]
\newtheorem{con}{{\heiti 条件}}[section]
\newtheorem{example}{{\heiti 例}}[section]

\renewcommand*{\proofname}{证明}

\newcommand{\nc}{\newcommand}
%定义特殊短语
\newcommand{\tbc}{\red{TO BE CONTINUED...}}
\newcommand{\opp}{\red{OPEN PROBLEMS}.~}
%定义颜色
\newcommand{\red}{\textcolor{red}}
% open questions
\newcommand{\blue}{\textcolor{blue}}
% suspicious result or derivation
\newcommand{\green}{\textcolor{green}}
\newcommand{\white}{\textcolor{white}}

%定义空心大写字母
\nc{\bbA}{\mathbb{A}} \nc{\bbB}{\mathbb{B}} \nc{\bbC}{\mathbb{C}}
\nc{\bbD}{\mathbb{D}} \nc{\bbE}{\mathbb{E}} \nc{\bbF}{\mathbb{F}}
\nc{\bbG}{\mathbb{G}} \nc{\bbH}{\mathbb{H}} \nc{\bbI}{\mathbb{I}}
\nc{\bbJ}{\mathbb{J}} \nc{\bbK}{\mathbb{K}} \nc{\bbL}{\mathbb{L}}
\nc{\bbM}{\mathbb{M}} \nc{\bbN}{\mathbb{N}} \nc{\bbO}{\mathbb{O}}
\nc{\bbP}{\mathbb{P}} \nc{\bbQ}{\mathbb{Q}} \nc{\bbR}{\mathbb{R}}
\nc{\bbS}{\mathbb{S}} \nc{\bbT}{\mathbb{T}} \nc{\bbU}{\mathbb{U}}
\nc{\bbV}{\mathbb{V}} \nc{\bbW}{\mathbb{W}} \nc{\bbX}{\mathbb{X}}
\nc{\bbZ}{\mathbb{Z}}
 
 %定义特殊符号
\newcommand{\bra}[1]{\langle#1|}
\newcommand{\ket}[1]{|#1\rangle}
\newcommand{\proj}[1]{| #1\rangle\!\langle #1 |}
\newcommand{\ketbra}[2]{|#1\rangle\!\langle#2|}
\newcommand{\braket}[2]{\langle#1|#2\rangle}
\newcommand{\wetw}[2]{|#1\rangle\wedge|#2\rangle}
\newcommand{\weth}[3]{|#1\rangle\wedge|#2\rangle\wedge|#3\rangle}
\newcommand{\wefo}[4]{|#1\rangle\wedge|#2\rangle\wedge|#3\rangle\wedge|#4\rangle}
\newcommand{\norm}[1]{\lVert#1\rVert}
\newcommand{\abs}[1]{|#1|}
%定义特殊运算符
\def\xr{X_\r}
\def\xrg{X_{\r^\G}}
\def\axr{\abs{X_\r}}
\def\axrg{\abs{X_{\r^\G}}}


\def\locc{\mathop{\rm LOCC}}
\def\lu{\mathop{\rm LU}}
\def\max{\mathop{\rm max}}
\def\min{\mathop{\rm min}}
\def\mspec{\mathop{\rm mspec}}
\def\oghz{\mathop{\overline{\ghz}}}
\def\per{\mathop{\rm per}}
\def\ppt{\mathop{\rm PPT}}
\def\pr{\mathop{\rm pr}}
%\bQ, \bR, \bZ denotes the set of rational, real and integer numbers.
\newcommand{\pp}[2]{{\partial #1\over\partial #2}}

\nc{\cA}{{\cal A}} \nc{\cB}{{\cal B}} \nc{\cC}{{\cal C}}
\nc{\cD}{{\cal D}} \nc{\cE}{{\cal E}} \nc{\cF}{{\cal F}}
\nc{\cG}{{\cal G}} \nc{\cH}{{\cal H}} \nc{\cI}{{\cal I}}
\nc{\cJ}{{\cal J}} \nc{\cK}{{\cal K}} \nc{\cL}{{\cal L}}
\nc{\cM}{{\cal M}} \nc{\cN}{{\cal N}} \nc{\cO}{{\cal O}}
\nc{\cP}{{\cal P}} \nc{\cQ}{{\cal Q}} \nc{\cR}{{\cal R}}
\nc{\cS}{{\cal S}} \nc{\cT}{{\cal T}} \nc{\cU}{{\cal U}}
\nc{\cV}{{\cal V}} \nc{\cW}{{\cal W}} \nc{\cX}{{\cal X}}
\nc{\cZ}{{\cal Z}}

%符号
\def\ra{\rightarrow}
\def\Ra{\Rightarrow}
\def\su{\subset}
\def\sue{\subseteq}
\def\sm{\setminus}
\def\we{\wedge}
\def\Ps{\Psi}
\def\Ph{\Phi}


\def\diag{\mathop{\rm diag}}
\def\dim{\mathop{\rm Dim}}
\def\epr{\mathop{\rm EPR}}
\def\ev{\mathop{\rm EV}}
\def\tr{\mathop{\rm Tr}}
\def\lin{\mathop{\rm span}}
\def\rank{\mathop{\rm rank}}

\def\ba{\begin{array}}
	\def\ea{\end{array}}
\def\be{\begin{equation}}
\def\ee{\end{equation}}
\def\bg{\begin{aligned}}
	\def\eg{\end{aligned}}


\DeclareMathOperator*{\argmax}{arg\,max}
\DeclareMathOperator*{\argmin}{arg\,min}
\DeclareMathOperator*{\expectation}{\mathbb{E}}
\DeclareMathOperator*{\minimize}{minimize}
% \DeclareMathOperator*{\prox}{prox}
\newcommand{\prox}{\mathbf{prox}}
\newcommand{\dom}{\operatorname{dom}}
\newcommand{\col}{\operatorname{col}}

\newcommand{\R}{\mathbb{R}}
\newcommand{\N}{\mathbb{N}}


\renewcommand{\theequation}{\thesection.\arabic{equation}}
\catcode`@=11 \@addtoreset{equation}{section} \catcode`@=12

\allowdisplaybreaks
%\setlength{\topskip}{0.3in}   %%%%%%表示正文和页眉的间距
%\baselineskip 24pt            %%%%%%表示正文行距
\makeatletter
\renewenvironment{thebibliography}[1]
%org     {\chapter*{\bibname
%org    \@mkboth{\MakeUppercase\bibname}{\MakeUppercase\bibname}}%
     {\def\chaptername{}\chapter*{\bibname\@mkboth{\MakeUppercase\bibname}{\MakeUppercase\bibname}}%                            !!!
      \list{\@biblabel{\@arabic\c@enumiv}}%
           {\settowidth\labelwidth{\@biblabel{#1}}%
            \leftmargin\labelwidth
            \advance\leftmargin\labelsep
            \@openbib@code
            \usecounter{enumiv}%
            \let\p@enumiv\@empty
            \renewcommand\theenumiv{\@arabic\c@enumiv}}%
      \small%                                               !!!
      \sloppy
      \clubpenalty4000
      \@clubpenalty \clubpenalty
      \widowpenalty4000%
      \sfcode`\.\@m}
     {\def\@noitemerr
       {\@latex@warning{Empty `thebibliography' environment}}%
      \endlist}
\makeatother
