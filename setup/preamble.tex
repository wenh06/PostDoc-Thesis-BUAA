%%%%%%%%%%%%%%%%%%%%%%%%%%%%%%%%%%%%%%%%%%%%%%%%%%%%%%%%
% packages
%%%%%%%%%%%%%%%%%%%%%%%%%%%%%%%%%%%%%%%%%%%%%%%%%%%%%%%%
\usepackage{xeCJK}
\usepackage{amsmath, amsthm, amscd, amssymb}
\usepackage{graphicx}
\usepackage{mathrsfs}
\usepackage{mathtools}
% \usepackage{citesort}
% \usepackage[numbers, sort&compress]{natbib}
\usepackage{anysize}
\marginsize{3cm}{3cm}{2.9cm}{2.8cm} \baselineskip 22pt
\usepackage[sf]{titlesec}
\usepackage{fancyhdr}
\usepackage{graphics}
\usepackage{graphicx}
\usepackage{subfigure}
% \usepackage{caption}
% \usepackage{ccaption}
\usepackage{tabularx}
\usepackage{multirow}
\usepackage{multicol}
% \usepackage{longtable}
% \usepackage{slashbox}
% \usepackage{supertabular}
\usepackage{float}
% \usepackage{diagbox}
% \usepackage{booktabs}
% \usepackage{cite}
\usepackage{bm}
\usepackage{arydshln}
% \usepackage{setspace}
\usepackage{boxedminipage}

\usepackage[ruled,linesnumbered,algosection,nofillcomment]{algorithm2e}
\renewcommand*{\algorithmcfname}{算法}
\renewcommand*{\algorithmautorefname}{算法}

\makeatletter
\renewcommand{\Indentp}[1]{%
  \advance\leftskip by #1
  \advance\skiptext by -#1
  \advance\skiprule by #1}%
\renewcommand{\Indp}{\algocf@adjustskipindent\Indentp{\algoskipindent}}
\renewcommand{\Indpp}{\Indentp{0.5em}}%
\renewcommand{\Indm}{\algocf@adjustskipindent\Indentp{-\algoskipindent}}
\renewcommand{\Indmm}{\Indentp{-0.5em}}%
\makeatother

\newcommand\mycommfont[1]{\footnotesize\ttfamily\textcolor{blue}{#1}}
\SetCommentSty{mycommfont}

\DontPrintSemicolon

\usepackage{tikz}
\usetikzlibrary{trees,arrows.meta,decorations.pathmorphing,decorations.pathreplacing,shapes,shapes.geometric,backgrounds,positioning,calc,tikzmark,external}

% 参考文献工具,加载biblatex宏包,
% 其后端backend使用biber,%标注(引用)样式citestyle,
% 著录样式 bibstyle都采用gb7714-2015样式,
% 两者相同时可以合并为一个选项style
% https://ctan.org/pkg/biblatex-gb7714-2015?lang=en
% https://www.overleaf.com/learn/latex/Articles/Getting_started_with_BibLaTeX
\usepackage[backend=biber,style=gb7714-2015]{biblatex}
\addbibresource[location=local]{references.bib}


%%%%%%%%%%%%%%%%%%%%%%%%%%%%%%%%%%%%%%%%%%%%%%%%%%%%%%%%
% Chinese fonts
%%%%%%%%%%%%%%%%%%%%%%%%%%%%%%%%%%%%%%%%%%%%%%%%%%%%%%%%

\setCJKmainfont[Path = fonts/, BoldFont = simhei.ttf]{simsun.ttc}
\setCJKsansfont[Path = fonts/, BoldFont = simhei.ttf]{simsun.ttc}
\setCJKmonofont[Path = fonts/, BoldFont = simhei.ttf]{simsun.ttc}

\newCJKfontfamily\kaishu[Path=fonts/]{simkai.ttf}
\newCJKfontfamily\songti[Path=fonts/]{simsun.ttf}
\newCJKfontfamily\heiti[Path=fonts/]{simhei.ttf}
\newCJKfontfamily\fangsong[Path=fonts/]{simfang.ttf}

%%%%%%%%%%%%%%%%%%%%%%%%%%%%%%%%%%%%%%%%%%%%%%%%%%%%%%%%
% miscelaneous settings
%%%%%%%%%%%%%%%%%%%%%%%%%%%%%%%%%%%%%%%%%%%%%%%%%%%%%%%%

% \numberwithin{algorithm}{section}%让算法按节编号!
% \floatname{algorithm}{算法}%将Algorithm替换为‘算法’

\newcommand{\citeu}[1]{$^{\mbox{\protect \scriptsize \cite{#1}}}$}
\newcommand{\eqabove}[1]{\stackrel{\mathclap{\normalfont\mbox{#1}}}{=}}
\newcommand{\neqabove}[1]{\stackrel{\mathclap{\normalfont\mbox{#1}}}{\neq}}

\newcommand{\p}{\partial}
%\newtheorem{exam}{\hei 算例}[chapter]
\newcommand\CJKprechaptername{第}
\newcommand\CJKchaptername{章}
\newcommand\CJKthechapter{\CJKnumber{\@arabic\c@chapter}}
\renewcommand{\tablename}{\kaishu 表}
\renewcommand{\figurename}{\kaishu 图}

\def\Def{~\overset{def}{=}~}  %def
\setlength{\unitlength}{1.2cm} %

\DeclareMathOperator*{\argmax}{arg\,max}
\DeclareMathOperator*{\argmin}{arg\,min}
\DeclareMathOperator*{\expectation}{\mathbb{E}}
\DeclareMathOperator*{\minimize}{minimize}
% \DeclareMathOperator*{\prox}{prox}
\newcommand{\prox}{\mathbf{prox}}

\newcommand{\R}{\mathbb{R}}

\renewcommand\arraystretch{1.2}
