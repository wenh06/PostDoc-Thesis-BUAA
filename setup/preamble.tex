%%%%%%%%%%%%%%%%%%%%%%%%%%%%%%%%%%%%%%%%%%%%%%%%%%%%%%%%
% packages
%%%%%%%%%%%%%%%%%%%%%%%%%%%%%%%%%%%%%%%%%%%%%%%%%%%%%%%%
\usepackage{xeCJK}
\usepackage{amsmath, amsthm, amscd, amssymb}
\usepackage{graphicx}
\usepackage{mathrsfs}
\usepackage{mathtools}
% \usepackage{citesort}
% \usepackage[numbers, sort&compress]{natbib}
\usepackage{anysize}
\marginsize{3cm}{3cm}{2.9cm}{2.8cm} \baselineskip 22pt
\usepackage[sf]{titlesec}
\usepackage{fancyhdr}
\usepackage{graphics}
\usepackage{graphicx}
\usepackage{subfigure}
% \usepackage{caption}
% \usepackage{ccaption}
\usepackage{tabularx}
\usepackage{multirow}
\usepackage{multicol}
% \usepackage{longtable}
% \usepackage{slashbox}
% \usepackage{supertabular}
\usepackage{float}
% \usepackage{diagbox}
% \usepackage{booktabs}
% \usepackage{cite}
\usepackage{bm}
\usepackage{arydshln}
\usepackage{setspace}
\usepackage{boxedminipage}
\usepackage[hidelinks,linktocpage]{hyperref}
\usepackage{imakeidx}
\makeindex[intoc,columns=2,columnsep=2em,title=名词索引]
\usepackage[nottoc, numbib]{tocbibind}
\usepackage{listings}
\usepackage{xcolor}
\usepackage{boldline,multirow}
\usepackage{threeparttable}

\definecolor{codegreen}{rgb}{0,0.6,0}
\definecolor{codegray}{rgb}{0.5,0.5,0.5}
\definecolor{codepurple}{rgb}{0.58,0,0.82}
\definecolor{backcolour}{rgb}{0.95,0.95,0.92}

\lstdefinestyle{mystyle}{
    backgroundcolor=\color{backcolour},   
    commentstyle=\color{codegreen},
    keywordstyle=\color{magenta},
    numberstyle=\tiny\color{codegray},
    stringstyle=\color{codepurple},
    basicstyle=\ttfamily\footnotesize,
    breakatwhitespace=false,         
    breaklines=true,                 
    captionpos=b,                    
    keepspaces=true,                 
    numbers=left,                    
    numbersep=5pt,                  
    showspaces=false,                
    showstringspaces=false,
    showtabs=false,                  
    tabsize=2
}

\lstset{style=mystyle}
\renewcommand{\lstlistingname}{代码}
\renewcommand{\lstlistlistingname}{List of \lstlistingname s}

\newcommand\YAMLcolonstyle{\color{red}\mdseries}
\newcommand\YAMLkeystyle{\color{black}\bfseries}
\newcommand\YAMLvaluestyle{\color{blue}\mdseries}

\makeatletter

% here is a macro expanding to the name of the language
% (handy if you decide to change it further down the road)
\newcommand\language@yaml{yaml}

\expandafter\expandafter\expandafter\lstdefinelanguage
\expandafter{\language@yaml}
{
  keywords={true,false,null,y,n},
  keywordstyle=\color{darkgray}\bfseries,
  basicstyle=\YAMLkeystyle,                                 % assuming a key comes first
  sensitive=false,
  comment=[l]{\#},
  morecomment=[s]{/*}{*/},
  commentstyle=\color{purple}\ttfamily,
  stringstyle=\YAMLvaluestyle\ttfamily,
  moredelim=[l][\color{orange}]{\&},
  moredelim=[l][\color{magenta}]{*},
  moredelim=**[il][\YAMLcolonstyle{:}\YAMLvaluestyle]{:},   % switch to value style at :
  morestring=[b]',
  morestring=[b]",
  literate =    {---}{{\ProcessThreeDashes}}3
                {>}{{\textcolor{red}\textgreater}}1     
                {|}{{\textcolor{red}\textbar}}1 
                {\ -\ }{{\mdseries\ -\ }}3,
}

% switch to key style at EOL
\lst@AddToHook{EveryLine}{\ifx\lst@language\language@yaml\YAMLkeystyle\fi}
\makeatother

\newcommand\ProcessThreeDashes{\llap{\color{cyan}\mdseries-{-}-}}

% from newtxmath
\DeclareFontFamily{U}{ntxmia}{}
\DeclareFontShape{U}{ntxmia}{m}{it}{<-> ntxmia }{}
\DeclareFontShape{U}{ntxmia}{b}{it}{<-> ntxbmia }{}
\DeclareSymbolFont{lettersA}{U}{ntxmia}{m}{it}
\SetSymbolFont{lettersA}{bold}{U}{ntxmia}{b}{it}

\AtBeginDocument{\let\mathbb\varmathbb}

\ExplSyntaxOn
\NewDocumentCommand{\varmathbb}{m}
 {
  \tl_map_inline:nn { #1 }
   {
    \use:c { varbb##1 }
   }
 }
\cs_new_protected:Nn \__mathbb_define:Nn
 {
  \DeclareMathSymbol{#1}{\mathord}{lettersA}{#2}
 }
\cs_generate_variant:Nn \__mathbb_define:Nn {ce}
\tl_map_inline:nn { ABCDEFGHIJKLMNOPQRSTUVWXYZ }
 {
  \__mathbb_define:ce { varbb#1 } { \int_eval:n { `#1+67 } }
 }
\tl_map_inline:nn { abcdefghijklmnopqrstuvwxyz }
 {
  \__mathbb_define:ce { varbb#1 } { \int_eval:n { `#1+61 } }
 }
\DeclareMathSymbol{\varbbimath}{\mathord}{lettersA}{'270}
\DeclareMathSymbol{\varbbjmath}{\mathord}{lettersA}{'271}
\ExplSyntaxOff

\usepackage[ruled,linesnumbered,algosection,nofillcomment]{algorithm2e}
\renewcommand*{\algorithmcfname}{算法}
\renewcommand*{\algorithmautorefname}{算法}

\makeatletter
\renewcommand{\Indentp}[1]{%
  \advance\leftskip by #1
  \advance\skiptext by -#1
  \advance\skiprule by #1}%
\renewcommand{\Indp}{\algocf@adjustskipindent\Indentp{\algoskipindent}}
\renewcommand{\Indpp}{\Indentp{0.5em}}%
\renewcommand{\Indm}{\algocf@adjustskipindent\Indentp{-\algoskipindent}}
\renewcommand{\Indmm}{\Indentp{-0.5em}}%
\makeatother

\newcommand\mycommfont[1]{\footnotesize\ttfamily\textcolor{blue}{#1}}
\SetCommentSty{mycommfont}

\DontPrintSemicolon

\usepackage{tikz}
\usetikzlibrary{trees,arrows.meta,decorations.pathmorphing,decorations.pathreplacing,shapes,shapes.geometric,backgrounds,positioning,calc,tikzmark,external}
\usepackage{pgfplots}
\pgfplotsset{compat=1.18}
\usepackage{environ}
\makeatletter
\newsavebox{\measure@tikzpicture}
\NewEnviron{scaletikzpicturetowidth}[1]{%
  \def\tikz@width{#1}%
  \def\tikzscale{1}\begin{lrbox}{\measure@tikzpicture}%
  \BODY
  \end{lrbox}%
  \pgfmathparse{#1/\wd\measure@tikzpicture}%
  \edef\tikzscale{\pgfmathresult}%
  \BODY
}
\makeatother

% 参考文献工具,加载biblatex宏包,
% 其后端backend使用biber,%标注(引用)样式citestyle,
% 著录样式 bibstyle都采用gb7714-2015样式,
% 两者相同时可以合并为一个选项style
% https://ctan.org/pkg/biblatex-gb7714-2015?lang=en
% https://www.overleaf.com/learn/latex/Articles/Getting_started_with_BibLaTeX
\usepackage[backend=biber,style=gb7714-2015]{biblatex}
\addbibresource[location=local]{references.bib}


%%%%%%%%%%%%%%%%%%%%%%%%%%%%%%%%%%%%%%%%%%%%%%%%%%%%%%%%
% Chinese fonts
%%%%%%%%%%%%%%%%%%%%%%%%%%%%%%%%%%%%%%%%%%%%%%%%%%%%%%%%

\setCJKmainfont[Path = fonts/, BoldFont = simhei.ttf]{simsun.ttc}
\setCJKsansfont[Path = fonts/, BoldFont = simhei.ttf]{simsun.ttc}
\setCJKmonofont[Path = fonts/, BoldFont = simhei.ttf]{simsun.ttc}

\newCJKfontfamily\kaishu[Path=fonts/]{simkai.ttf}
\newCJKfontfamily\songti[Path=fonts/]{simsun.ttf}
\newCJKfontfamily\heiti[Path=fonts/]{simhei.ttf}
\newCJKfontfamily\fangsong[Path=fonts/]{simfang.ttf}

%%%%%%%%%%%%%%%%%%%%%%%%%%%%%%%%%%%%%%%%%%%%%%%%%%%%%%%%
% miscelaneous settings
%%%%%%%%%%%%%%%%%%%%%%%%%%%%%%%%%%%%%%%%%%%%%%%%%%%%%%%%

% \numberwithin{algorithm}{section}%让算法按节编号!
% \floatname{algorithm}{算法}%将Algorithm替换为‘算法’

\newcommand{\citeu}[1]{$^{\mbox{\protect \scriptsize \cite{#1}}}$}
\newcommand{\eqabove}[1]{\stackrel{\mathclap{\normalfont\mbox{#1}}}{=}}
\newcommand{\neqabove}[1]{\stackrel{\mathclap{\normalfont\mbox{#1}}}{\neq}}

\newcommand{\p}{\partial}
%\newtheorem{exam}{\hei 算例}[chapter]
\newcommand\CJKprechaptername{第}
\newcommand\CJKchaptername{章}
\newcommand\CJKthechapter{\CJKnumber{\@arabic\c@chapter}}
\renewcommand{\tablename}{\kaishu 表}
\renewcommand{\figurename}{\kaishu 图}

\def\Def{~\overset{def}{=}~}  %def
\setlength{\unitlength}{1.2cm} %

\renewcommand\arraystretch{1.2}
