%\mbox{}\newpage
\chapter*{符~号~说~明}
\headheight=15.24pt%5mm
\addcontentsline{toc}{chapter}{\numberline {{\heiti 符~号~说~明}}~~~}
\addcontentsline{toe}{chapter}{{{\bf Notation}}\numberline~}
\markboth{符~号~说~明}{北京航空航天大学文豪博士后研究工作报告}
%%%%%%%%%%%%%%%%%%%%%%%%%%%%%%%%%%%%%%%%%%%%%%%%%%%%%%%%%5
这里给出本文常用的一些符号, 未包含其中的符号将在文中用到时具体说明.
\begin{equation*}
  \begin{array}{lll}
\mathbb{R}&\qquad\qquad\qquad\qquad\qquad&\text{实数集}\\
\mathbb{C}&&\text{复数集}\\
\mathbb{R}^{n} &&\text{实~$n$ 维列向量集} \\
\mathbb{C}^{n} &&\text{复~$n$ 维列向量集} \\
\mathbb{R}^{m\times n} && \text{实~$m\times n$ 矩阵的集合} \\
\mathbb{C}^{m\times n} && \text{复~$m\times n$ 矩阵的集合} \\
%\mathbf{0}&& \text{零向量或零矩阵}\\
%\nabla &&\text{梯度算子} \\
%\nabla \cdot  &&\text{散度算子} \\
%%\nabla \times  &&\text{旋度算子} \\
%\Delta  &&\text{拉普拉斯算子} \\
\mathrm{i}&&\text{虚单位}\\
\overline{\theta} && \text{复数~$\theta$~的共轭}\\
|\cdot| &&\text{实数的绝对值或复数的模}\\
I_n~\text{或}~I &&\text{($n$~阶)单位矩阵}\\
O_n~\text{或}~O &&\text{($n$~阶)零矩阵}\\
A^\mathrm{T} &&\text{矩阵~$A$~的转置矩阵}\\
A^* &&\text{矩阵~$A$~的共轭转置矩阵}\\
A^{-1}&&  \text{矩阵~$A$~的逆矩阵}\\
\sigma(A)&&\text{矩阵~$A$~的谱集}\\
\rho(A) &&\text{矩阵~$A$~的谱半径}\\
\nu(A) &&\text{矩阵~$A$~的拟谱半径}\\
%\lambda_{\min}(A)&&\text{矩阵~$A$~的最小特征值}\\
%\lambda_{\min}^+(A)&&\text{矩阵~$A$~的最小正特征值}\\
%\lambda_{\max}(A)&&\text{矩阵~$A$~的最大特征值}\\
\mathrm{rank}(A)&&\text{矩阵~$A$~的秩}\\
\mathrm{null}(A)&&\text{矩阵~$A$~的零空间}\\
\mathrm{range}(A)&&\text{矩阵~$A$~的值域}\\
\mathrm{index}(A)&&\text{矩阵~$A$~的指标}\\
%\mathrm{tr}(A)&&\text{矩阵~$A$~的迹}\\
A\otimes B&&\text{矩阵~$A$~和~$B$~的\,Kronecker\,积}\\
\|\cdot\|_2&&\text{向量或矩阵的~$2$-范数}\\
%\|\cdot\|_A$$ && \text{向量的~$A$-范数}\\
%\|\cdot\|_F&&\text{矩阵的\,Frobenius\,范数}\\
A\succ O&&\text{矩阵\,$A$\,为对称正定矩阵}\\
\mathrm{span}\{x_1,\cdots,x_n\}&&\text{向量~$x_1,\cdots,x_n$~张成的线性空间}
\end{array}
\end{equation*}
\nopagebreak[4]
