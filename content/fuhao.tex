%\mbox{}\newpage
\chapter*{符~号~说~明}
\headheight=15.24pt%5mm
\addcontentsline{toc}{chapter}{\numberline {{\heiti 符~号~说~明}}~~~}
\addcontentsline{toe}{chapter}{{{\bf Notation}}\numberline~}
\markboth{符~号~说~明}{北京航空航天大学文豪博士后研究工作报告}
%%%%%%%%%%%%%%%%%%%%%%%%%%%%%%%%%%%%%%%%%%%%%%%%%%%%%%%%%5
这里给出本文常用的一些符号, 未包含其中的符号将在文中用到时具体说明.
\begin{equation*}
\begin{array}{lll}
~\N & \qquad\qquad\qquad\qquad\qquad & \text{自然数集(包括$0$)}\\
~\R & & \text{实数集}\\
~\R^{d} & & \text{实$d$维列向量空间} \\
% ~\R^{m\times n} & & \text{实$m \times n$矩阵的集合} \\
~\Pi_{\Omega} & & \text{向凸集$\Omega \in \R^d$投影映射} \\
~[K] & & \text{集合$\{ 1, 2, \ldots, K \},$ $K \in \N_{> 0}$} \\
~\theta & & \text{模型的参数,$\in \R^d$} \\
~\col(\theta_1, \cdots, \theta_K) & & \text{列向量$\theta_1, \ldots, \theta_K$按顺序竖排在一起形成的列向量} \\
~\operatorname{dom}(f) & & \text{函数$f$的定义域,即$\{\theta \in \R^d ~|~ f(\theta) < +\infty\}$} \\
~\nabla f(\cdot) & & \text{函数$f$的梯度} \\
~\partial f(\cdot) & & \text{函数$f$的次梯度} \\
~\mathcal{D}(x, y) & & \text{数据的分布(Distribution),} \\
& & \text{$x$是特征(Feature),$y$是标签(Label)} \\
~I_n~\text{或}~I & & \text{($n$阶)单位矩阵} \\
~O_n~\text{或}~O & & \text{($n$阶)零矩阵} \\
~A^\mathrm{T} & & \text{矩阵$A$的转置矩阵} \\
~A^{-1} & & \text{矩阵$A$的逆矩阵} \\
% \sigma(A) & & \text{矩阵$A$的谱集} \\
% \rho(A) & & \text{矩阵$A$的谱半径} \\
% \lambda_{\min}(A) & & \text{矩阵$A$的最小特征值} \\
% \lambda_{\min}^+(A) & & \text{矩阵$A$的最小正特征值} \\
% \lambda_{\max}(A) & & \text{矩阵$A$的最大特征值} \\
% \mathrm{rank}(A) & & \text{矩阵$A$的秩} \\
% \mathrm{null}(A) & & \text{矩阵$A$的零空间} \\
% \mathrm{range}(A) & & \text{矩阵$A$的值域} \\
% \mathrm{index}(A) & & \text{矩阵$A$的指标} \\
~\lVert \cdot \rVert_p & & \text{向量或矩阵的$p$-范数,$0 \leqslant p \leqslant +\infty,$} \\
& & \text{当$p$不写时,表示向量或矩阵的$2$-范数} \\
~\lVert \cdot \rVert_F & & \text{矩阵的Frobenius范数} \\
~\expectation[\cdot] & & \text{数学期望} \\
\end{array}
\end{equation*}
\nopagebreak[4]
