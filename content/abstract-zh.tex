\chapter*{摘~~~要}
\headheight=15.24pt%5mm
\markboth{摘~~~要}{摘~~~要}
\addcontentsline{toc}{chapter}{{{\hei 摘~要}}\numberline ~}
\addcontentsline{toe}{chapter}{{{Abstract in Chinese}}\numberline ~}

\pagenumbering{Roman}\setcounter{page}{1}

% finished

联邦学习是伴随着去中心化、保护隐私等需求而兴起的一种人工智能的技术。它是一种全新的机器学习范式,适用于有中心节点居中协调,多个子节点进行协作参与,进行联合建模的场景。在这种场景下,联邦学习的方法能够在每个参与方不暴露己方拥有的数据的情况下,完成协同建模的任务。目前,联邦学习已成为了机器学习、人工智能领域的一个活跃的分支领域。

本文主要关注的是联邦学习中的优化算法及其在个性化联邦学习当中的应用。本文首先详细介绍联邦学习相关的多种优化算法,特别是分解算法。分解算法基本思想是将大规模的问题分解成一系列小规模的问题,将每个子问题分开处理,使得迭代算法中的子问题更容易求解或者更容易并行化。典型的分解算法包括了算子分裂法、交替方向乘子法等。目前已经有了一系列的工作,展现了分解算法在联邦学习问题研究中的独特优势。

随后,本文介绍个性化联邦学习特有的一些算法,并将分解算法应用于个性化联邦学习问题的研究。个性化联邦学习是处理具有复杂数据分布的联邦学习问题的一种方法,在协同训练公共模型的同时,允许参与方得到保留个性化特征、稍有差异的本地模型,提升了相关复杂场景下联邦学习的适用性。个性化联邦学习在问题建模、算法设计等方面仍有大量未解决的问题,本文将在这些问题上进行深入探讨,形成一些独到的研究。

最后,文本将介绍为联邦学习算法效果验证设计的一套仿真系统,并用此仿真系统进行数值实验,并与已有的一些算法形成对比,验证本文提出算法的效果。

\par
\bigskip

{\song \textbf{关键词}: 联邦学习, 协同建模, 分解算法, 算子分裂法, 模型个性化}

% \newpage
% ~~~\vspace{1em}
% \thispagestyle{empty}
