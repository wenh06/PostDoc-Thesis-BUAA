%\mbox{}\newpage
\chapter{\texorpdfstring{\hspace{-1mm}\bf 未来工作展望}{未来工作展望}}
\label{chap7}
\addcontentsline{toe}{chapter}{{{\bf Chapter \thechapter \ 
Directions for Future Work }}\numberline\,}
\markboth{第 \thechapter 章\ \ 
未来工作展望}{北京航空航天大学博士后研究工作报告}

%%%%%%%%%%%%%%%%%%%%%%%%%%%%%%%%%%%%%%%%%%%%%%%%%%%%%%%%%%%

% almost finished

联邦学习是机器学习领域内非常年轻的分支,这一分支内各个方向的研究正在蓬勃地发展,从算法理论到实际问题的应用,不断地有新的问题被提出,新的工具被引入。笔者所研究的联邦学习算法的理论,特别是个性化联邦学习问题中的算子分裂理论及算法,是联邦学习领域内的热门研究方向。在本报告的\S\ref{sec:chap3-pfl-os}~节中,笔者列出了这一方向一些值得继续深入探究的问题,这也是笔者在今后的研究工作中,会着重关注的问题。笔者实现的联邦学习仿真系统\texttt{fl-sim}将会是相关研究工作的得力工具。

联邦学习的激励机制 (Incentive Mechanism)\index{激励机制, Incentive Mechanism} 是笔者在博士后这一阶段的研究工作后期逐渐开始关注的内容。联邦学习兴起于实际问题,其理论与算法最终也需要落地应用到实际问题中去。联邦学习的参与方本质上是一些相对独立的实体,参与联邦学习训练获得模型效果提升这一效益的同时,也要承担相应的计算成本,同时也为利益不相关的其他参与方做出了贡献。联邦学习参与方掌握的数据越多,所需要承担的计算成本以及对其他参与方做出的模型效果提升的贡献往往也越大,在没有激励机制的情况下参与联邦学习训练的积极性往往不高\cite{kairouz2019advances_fl, Zhan_2021_incentive_fl},这是阻碍联邦学习实际应用的一大问题。因此,合理、安全、高效的激励机制的设计\cite{Kang_2019_incentive_fl, Zhan_2020_incentive_fl}对于联邦学习的应用推广至关重要。笔者也将在日后的研究工作中,重点关注这一问题。

区块链 (Blockchain)\index{区块链, Blockchain} 借由密码学与共识机制等技术的点对点网络系统\cite{Swan_2015_Blockchain}。这一技术首先诞生自数字加密货币 (Digital Cryptocurrency) 比特币BitCoin,随后在金融、制造、农业等多个领域获得应用。其去中心化、隐私保护以及结果可信等特点与联邦学习非常契合,因此有不少将二者结合的研究工作\cite{kairouz2019advances_fl, Wang_2022_Blockchain_fl, Li_2021_Blockchain_fl},特别是之前提到的联邦学习激励机制的问题\cite{Liu_2020_FedCoin, Wang_2022_Blockchain_fl}。因此,区块链与联邦学习结合的交叉研究领域,也是笔者计划深入研究的领域。

笔者在完成博士后这一阶段的研究工作之后,即将进入农业相关的高等院校工作。智慧农业是联邦学习乃至区块链技术很好的应用落地的场景,也是国家``十四五''规划重点扶持与推进的领域之一。将联邦学习的技术成功应用到智慧农业的多个领域,将是笔者的长期工作目标。
