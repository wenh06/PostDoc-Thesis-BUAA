%\mbox{}\newpage
\chapter{\hspace{-1mm}\bf 联邦学习仿真系统的设计与实现}
\label{chap5}
\addcontentsline{toe}{chapter}{{{\bf Chapter 5\ \
 Design and Implementation of a Federated Learning Simulation System}}\numberline\,}
\markboth{第\,5\,章\ \
联邦学习仿真软件设计与实现}{北京航空航天大学博士后研究工作报告}

\newcommand{\urlgithub}{\url{https://github.com/wenh06/fl-sim}}
\newcommand{\urlgitee}{\url{https://gitee.com/wenh06/fl-sim}}

%%%%%%%%%%%%%%%%%%%%%%%%%%%%%%%%%%%%%%%%%%%%%%%%%%%%%%%%%%%

\section{仿真系统设计}
\addcontentsline{toe}{section}{{5.1\ \ Design of the Simulation System}\numberline\,}
\label{sec:chap5-design}

% NOT finished
% NOT indexed

本文实现的联邦学习仿真系统采用Python语言编写,整体结构借鉴了FedML\footnote{\url{https://github.com/FedML-AI/FedML}}\cite{he_2020_fedml},\texttt{FedProx}\footnote{\url{https://github.com/litian96/FedProx}}\cite{sahu2018fedprox}以及pFL-Bench\cite{chen_2022_pfl_bench}。实现的基本思路是基于配置文件 (或者Python类),将数据集读取、模型参数初始化、节点分配等工作自动化。同时将节点与算法解耦合,节点需要执行的通用操作,例如通讯、获取模型参数等进行统一实现,算法相关的操作通过预留接口的方式完成。这样,利用这一套仿真系统,可以专注于联邦学习优化算法的实现与验证工作。

本文实现的联邦学习仿真系统主要包含以下几个模块
\begin{itemize}
    \item \texttt{data\_processing}: 待写。。。。
    \item \texttt{models}: 待写。。。。
    \item \texttt{optimizers}: 待写。。。。
    \item \texttt{algorithms}: 待写。。。。
\end{itemize}

我们定义了一个抽象基类 (Abstract Base Class, ABC)\index{抽象基类, Abstract Base Class, ABC} \texttt{Node}作为子节点的类\texttt{Client}以及中心节点\texttt{Server}的公共基类,实现了所有类型的节点都要执行的公有操作:
\begin{itemize}
    \item \texttt{get\_detached\_model\_parameters}:获取当前节点分离形式 (即从PyTorch的计算图 (Computation Graph) \index{计算图, Computation Graph}) 的模型参数。
    \item \texttt{get\_gradients}:获取当前被训练模型 (或者说目标函数) 的梯度,或者梯度的范数。
\end{itemize}
同时,以抽象方法 (Python abstractmethod) 的形式,约定所有类型的节点都必须实现的操作,包括
\begin{itemize}
    \item \texttt{communicate}: 节点间通信的操作。
    \item \texttt{update}: 节点执行迭代,更新节点上各种类型参数 (主要是更新正在训练的模型参数) 的操作。
    \item \texttt{\_post\_init}, \texttt{required\_config\_fields}: 检查初始化是否合法相关的操作。
\end{itemize}

中心节点以及子节点分别继承\texttt{Node}类,实现相应类型节点特有的一些操作,例如中心节点\texttt{Server}实现的通用操作有
\begin{itemize}
    \item \texttt{\_setup\_clients}, \texttt{\_allocate\_devices}: 为子节点分配训练数据,初始化模型参数,分配内存或显存空间等
    \item \texttt{\_sample\_clients}: 待写。。。。
    \item \texttt{add\_parameters}, \texttt{avg\_parameters}: 待写。。。。
\end{itemize}

本文实现的联邦学习仿真系统的总体架构图可以总结为图 待写。。。。相关的代码地址为:
\begin{itemize}
    \item GitHub: \urlgithub
    \item gitee: \urlgitee
\end{itemize}

\section{联邦学习数据集}
\addcontentsline{toe}{section}{{5.2\ \ Datasets for Federated Learning}\numberline\,}
\label{sec:chap5-datasets}

% NOT finished
% NOT indexed

待写。。。。

\section{待写}
\addcontentsline{toe}{section}{{5.3\ \ To write}\numberline\,}
% \label{}

% NOT finished
% NOT indexed

待写。。。。
