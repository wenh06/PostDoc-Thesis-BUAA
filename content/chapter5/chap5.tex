%\mbox{}\newpage
\chapter{\hspace{-1mm}\bf 联邦学习仿真系统的设计与实现}
\label{chap5}
\addcontentsline{toe}{chapter}{{{\bf Chapter 5\ \
 Design and Implementation of a Federated Learning Simulation System}}\numberline\,}
\markboth{第\,5\,章\ \
联邦学习仿真软件设计与实现}{北京航空航天大学博士后研究工作报告}

\newcommand{\urlgithub}{\url{https://github.com/wenh06/pFedSplit}}
\newcommand{\urlgitee}{\url{https://gitee.com/wenh06/pFedSplit}}

%%%%%%%%%%%%%%%%%%%%%%%%%%%%%%%%%%%%%%%%%%%%%%%%%%%%%%%%%%%

\section{仿真系统设计}
\addcontentsline{toe}{section}{{5.1\ \ Design of the Simulation System}\numberline\,}
\label{sec:chap5-design}

% NOT finished
% NOT indexed

本文实现的联邦学习仿真系统采用Python语言编写,整体结构借鉴了FedML\footnote{\url{https://github.com/FedML-AI/FedML}}\cite{he_2020_fedml}以及\texttt{FedProx}\footnote{\url{https://github.com/litian96/FedProx}}\cite{sahu2018fedprox}。实现的基本思路是基于配置文件 (或者Python类),将数据集读取、模型参数初始化、节点分配等工作自动化。同时将节点与算法解耦合,节点需要执行的通用操作,例如通讯、获取模型参数等进行统一实现,算法相关的操作通过预留接口的方式完成。这样,利用这一套仿真系统,可以专注于联邦学习优化算法的实现与验证工作。

本文实现的联邦学习仿真系统主要包含以下几个模块
\begin{itemize}
    \item \texttt{data\_processing}: 待写。。。。
    \item \texttt{models}: 待写。。。。
    \item \texttt{optimizers}: 待写。。。。
    \item \texttt{algorithms}: 待写。。。。
\end{itemize}

我们定义了一个抽象基类 (Abstract Base Class, ABC)\index{抽象基类, Abstract Base Class, ABC} \texttt{Node}作为子节点的类\texttt{Client}以及中心节点\texttt{Server}的公共基类,在其中约定所有类型的节点都必须实现的操作,包括
\begin{itemize}
    \item \texttt{communicate}: 节点间通信的操作
    \item \texttt{update}: 节点执行迭代,更新节点上各种类型参数 (主要是更新正在训练的模型参数) 的操作
    \item \texttt{\_post\_init}, \texttt{required\_config\_fields}: 检查初始化是否合法相关的操作
    \item \texttt{get\_detached\_model\_parameters}: 用于获取当前模型参数
\end{itemize}
\texttt{Node}的实现具体可见代码~\ref{lst:node}。

\begin{lstlisting}[language=Python, caption=基类\texttt{Node}的Python代码,label={lst:node}]
from abc import ABC, abstractmethod

from torch_ecg.utils import ReprMixin


class Node(ReprMixin, ABC):
    """An abstract base class for the server and client nodes."""

    __name__ = "Node"

    @abstractmethod
    def communicate(self, target: "Node") -> None:
        """Communicate with the target node.

        The current node communicates model parameters, gradients, etc. to `target` node.
        For example, a client node communicates its local model parameters to server node via

        .. code-block:: python

            target._received_messages.append(
                ClientMessage(
                    {
                        "client_id": self.client_id,
                        "parameters": self.get_detached_model_parameters(),
                        "train_samples": self.config.num_epochs * self.config.num_steps * self.config.batch_size,
                        "metrics": self._metrics,
                    }
                )
            )

        For a server node, global model parameters are communicated to clients via

        .. code-block:: python

            target._received_messages = {"parameters": self.get_detached_model_parameters()}

        """
        raise NotImplementedError

    @abstractmethod
    def update(self) -> None:
        """Update model parameters, gradients, etc.
        according to `self._reveived_messages`.
        """
        raise NotImplementedError

    def _post_init(self) -> None:
        """Check if all required field in the config are set."""
        assert all(
            [hasattr(self.config, k) for k in self.required_config_fields]
        ), f"missing required config fields: {list(set(self.required_config_fields) - set(self.config.__dict__))}"

    @property
    @abstractmethod
    def required_config_fields(self) -> List[str]:
        """The list of required fields in the config."""
        raise NotImplementedError

    def get_detached_model_parameters(self) -> List[Tensor]:
        """Get the detached model parameters."""
        return [p.detach().clone() for p in self.model.parameters()]
\end{lstlisting}

中心节点以及子节点分别继承\texttt{Node}类,实现相应类型节点特有的一些操作,例如中心节点\texttt{Server}实现的通用操作有
\begin{itemize}
    \item \texttt{\_setup\_clients}, \texttt{\_allocate\_devices}: 为子节点分配训练数据,初始化模型参数,分配内存或显存空间等
    \item \texttt{\_sample\_clients}: 待写。。。。
    \item \texttt{add\_parameters}, \texttt{avg\_parameters}: 待写。。。。
\end{itemize}

本文实现的联邦学习仿真系统的总体架构图可以总结为图 待写。。。。相关的代码地址为:
\begin{itemize}
    \item GitHub: \urlgithub
    \item gitee: \urlgitee
\end{itemize}

\section{联邦学习数据集}
\addcontentsline{toe}{section}{{5.2\ \ Datasets for Federated Learning}\numberline\,}
\label{sec:chap5-datasets}

% NOT finished
% NOT indexed

待写。。。。

\section{待写}
\addcontentsline{toe}{section}{{5.3\ \ To write}\numberline\,}
% \label{}

% NOT finished
% NOT indexed

待写。。。。
