\section{可视化子系统}
\addcontentsline{toe}{section}{{\currentchapter .3\ \ Visualization Subsystem}\numberline\,}
\label{sec:chap5-viz}

% almost finished
% indexed

联邦学习,乃至更广范围的机器学习领域,过程与结果的可视化是一个比较重要的方面,例如著名的深度学习软件包TensorFlow\cite{tensorflow}有配套的可视化工具TensorBoard。本文实现的联邦学习仿真系统包含了为联邦学习量身定制的试验结果可视化面板系统\texttt{Panel}以及与之配套的试验日志系统。日志系统将试验过程中的模型评测数值同时以结构化的JSON文件,以及符合人类阅读习惯的文本文件的形式进行存储。可视化的面板系统\texttt{Panel}基于ipywidgets\footnote{\url{https://github.com/jupyter-widgets/ipywidgets}}开发,具备如下主要功能与特性
\begin{itemize}
    \item 自动从日志文件夹中搜索、列出所有完整试验的日志文件。
    \item 自动解析日志文件,完成模型评测数值曲线绘制。
    \item 支持绘制曲线 (图形) 平滑度、字体以及字体大小的动态调整。
    \item 支持绘制的图片以~PDF/SVG/PNG/JPEG/PS~等格式保存。
    \item 支持关键字形式的数值曲线合并,合并为均值曲线$\pm$误差界 (Error Bound) 的形式。误差界有4种选择,分别是标准差 (Standard Deviation, STD),样本均值的估计标准误差 (Standard Error of the Mean, SEM),四分位数 (Quartile)以及四分位距 (Interquartile Range, IQR).
\end{itemize}

\begin{figure}[ht]
    \centering
    \includegraphics[width=\textwidth]{figures/panel-init.png}
    \caption{可视化的面板系统\texttt{Panel}的初始化界面}
    \label{fig:panel-init}
\end{figure}

图\ref{fig:panel-init}~是可视化的面板系统\texttt{Panel}的初始化界面。图\ref{fig:panel-init}~是可视化的面板系统\texttt{Panel}的使用示例,图中为使用CIFAR10数据集对\texttt{IFCA}算法\cite{Ghosh_2022_cfl}进行10轮试验 (5个不同的随机种子$\times$2种不同的数据增强组合),同时以\texttt{FedAvg}算法以及局部训练作为对比方法的试验结果 (模型在验证集上的准确率)。可以清楚地看到\texttt{IFCA}算法相较于\texttt{FedAvg}算法在这一组试验上有明显的数值优势。

\begin{figure}[ht]
    \centering
    \includegraphics[width=\textwidth]{figures/panel-in-use.png}
    \caption{可视化的面板系统\texttt{Panel}的使用示例}
    \label{fig:panel-in-use}
\end{figure}
