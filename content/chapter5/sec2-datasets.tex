\section{联邦学习数据集}
\addcontentsline{toe}{section}{{5.2\ \ Datasets for Federated Learning}\numberline\,}
\label{sec:chap5-datasets}

% NOT finished
% NOT indexed

正如本文\S\ref{sec:chap1-fl-applications}~中提到的,联邦学习所涉及、处理的实际数据往往具有强烈的统计异质性,因此用于联邦学习仿真试验的试验数据集需要以多种方式、从多个角度尽量模拟、贴近这种统计异质性。比较幸运的是,在本章\S\ref{sec:chap5-design}中提到的一些联邦学习代码框架,包括FedML\cite{he_2020_fedml}, LEAF\cite{caldas2018_leaf},以及\texttt{FedProx}\cite{sahu2018fedprox}, \texttt{IFCA}\cite{Ghosh_2022_cfl}等联邦学习算法文献,已经将多个数据集依照联邦学习场景的特点进行了相应的适配以及标准化,方便进行联邦学习算法的对比研究,以及效果复现等工作。

本文实现的联邦学习仿真系统将其中常用的一些联邦学习数据集封装、集成到了\texttt{data\_processing}模块中,并添加了一系列方便使用的功能,这一点已经在上一节\S\ref{sec:chap5-design}~中已经介绍过了。表\ref{tab:datasets}~列出了\texttt{data\_processing}模块中集成的联邦学习数据集的主要信息。

\begin{table}[htbp]
\centering
\begin{threeparttable}[b]
\begin{tabular}{|c|c|c|c|c|}
\hlineB{3.5}
数据集名称 & 规模 & 默认节点数目 & 任务 & 样本类型 \\
\hline \hline
MNIST\tnote{$\ast$} & 60000 & 1000 & 图像分类 & $28\times 28$的单通道灰度图像 \\
EMNIST\tnote{$\ast$} & 749068 & 3400 & 图像分类 & $28\times 28$的单通道灰度图像 \\
CIFAR10/100 & 60000 & 500 & 图像分类 & $32\times 32$的RGB3通道图像 \\
Shakespeare & 18424 & 715 & 下一字符预测 & 文本 \\
Sent140 & 40783 & 715 & 文本情感分类 & 文本 \\
Synthetic($\alpha, \beta$)\tnote{$\dagger$} & N/A & N/A & 分类 & 随机生成的高维向量 \\
\hlineB{3.5}
\end{tabular}
\begin{tablenotes}
\item[$\ast$] {\smaller 这两个数据集还有经过筛选\cite{sahu2018fedprox}的规模更小的数据子集,也被本文实现的联邦学习仿真系统所包含。}
\item[$\dagger$] {\smaller 参数$\alpha, \beta$是两个独立的均值为$0$的正态分布的标准差,用于模拟节点内以及节点间的数据分布差异。}
\end{tablenotes}
\caption{本文开发的联邦学习仿真系统内置的数据集}
\label{tab:datasets}
\end{threeparttable}
\end{table}


\begin{figure}[H]
\centering
\includegraphics[width=\textwidth]{figures/fedproxfemnist.pdf}
\caption{待写}
\label{fig:fedproxfemnist-experiment}
\end{figure}

待写。。。。
