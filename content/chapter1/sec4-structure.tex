\section{本文结构}
\addcontentsline{toe}{section}{{1.4\ \ Structure of the Thesis}\numberline\,}
\label{sec:chap1-structure}

% almost finished

本文在第\ref{chap2}~章会先对联邦学习中已有的优化算法进行回顾,并会着重介绍其中的原始对偶(Primal-Dual)算法以及算子分裂(Operator Splitting)算法。在接下来的第\ref{chap3}~章介绍联邦学习中的模型个性化的概念,介绍其产生的动机,已有的一些方法,并引入算子分裂法来处理这个问题。为了方便联邦学习算法的快速验证,笔者使用Python语言实现了一套联邦学习的仿真软件。这套软件的设计原则与模块会在本文的第\ref{chap4}~章中进行介绍。在接下来的第\ref{chap5}~章中,本文利用这套仿真软件进行数值实验,验证本文提出的联邦学习算法的有效性以及优越性。
