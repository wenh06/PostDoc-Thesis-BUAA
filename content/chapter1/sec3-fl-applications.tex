\section{联邦学习的应用与发展前景}
\addcontentsline{toe}{section}{{1.3\ \ Applications and Future Perspective of Federal Learning}\numberline\,}
\label{sec:chap1-fl-applications}

尽管联邦学习这一概念被提出\cite{mcmahan2017fed_avg}的时间不长,但是联邦学习已经有了不少实际的应用。我们这里仅列举一些比较重要的例子。
\begin{itemize}
    \item 首先是金融领域,联邦学习在财会\cite{Schreyer_2022_fl_audits},信贷风控\cite{Imteaj_2022_fl},反欺诈\cite{Lv_2021_fl},防范金融犯罪\cite{Toyotaro_2019_fl}等各个方面都有了实际的应用。这其中最引人注目的当属我国微众银行在这一领域的一系列前沿以及开创性的研究\cite{Yang_2019_VFL, liu_2020_transfer_fl, vfl},并以真正落地形成工业级产品的联邦学习应用框架FATE\cite{liu_2021_fate_fl}.
    \item 联邦学习在智慧医疗领域也有颇多成功应用的案例\cite{rauniyar2022_fl_medical, Antunes_2022_fl_healthcare}。从电子健康档案(Electronic Health Records, EHR)处理\cite{Brisimi_2018_fl_ehr},医学影像处理\cite{Li_2020_fl_mri},到新冠(COVID-19)等流行病的研究\cite{Dayan_2021_fl_covid},都有联邦学习的应用案例。医学因为其数据隐私性的严格要求,同时又有市场的巨大需求驱动,是联邦学习非常适宜的应用领域,同时也可能是相关研究最活跃、应用落地最多的领域之一。
    \item 移动设备、物联网领域:这是联邦学习发轫\cite{mcmahan2017fed_avg}的领域,也是大规模跨设备(Cross-Device)场景最常见的领域,应用也从最一开始的安卓设备键盘辅助输入扩展到了移动设备的语音处理\cite{Leroy_2019_fl_ks},人体活动识别(Human Activity Recognition)\cite{Sozinov_2018_fl_human},乃至人口流动预测(Human Mobility Prediction)\cite{feng_2020_fl_pmf}等问题。因为隐私性以及模型个性化(Model Personalization)等要求,很多时候联邦学习甚至是解决问题的唯一选择。
\end{itemize}

除了以上列举的一些领域外,联邦学习的应用还逐渐扩大到了智慧城市(Smart City)\cite{Zheng_2021_fl_smart_city}纳米材料\cite{Huang_2022_fl_physics}等全新的领域。可以说,联邦学习的应用场景相较于它被提出时已经得到了极大的丰富,用于仿真试验的数据集也有了很多积累。同时,也涌现出了像FedML\cite{he_2020_fedml},TensorFlow Federated\cite{tensorflow},FedSim\cite{wu_2021_fedsim},PySyft\cite{ryffel_2018_pysyft},以及之前提到的FATE\cite{liu_2021_fate_fl}等多个有不同侧重点的联邦学习软件包,为联邦学习的研究与应用落地,都提供了极大的便利。然而目前联邦学习在理论以及实际应用上,还有不少问题没有完善的解决,仍然面临着诸多的挑战\cite{kairouz2019advances_fl, Li_2020_fl_challenges}。综述性的文章\parencite{Li_2020_fl_challenges}

待写。。。。
