%\mbox{}\newpage
\chapter{\hspace{-1mm}\bf 联邦学习中的优化算法}
\label{chap2}
\addcontentsline{toe}{chapter}{{{\bf Chapter 2\ \
Optimization Algorithms in Federated Learning }}\numberline\,}
\markboth{第\,2\,章\ \
待写}{北京航空航天大学博士后研究工作报告}

%%%%%%%%%%%%%%%%%%%%%%%%%%%%%%%%%%%%%%%%%%%%%%%%%%%%%%%%%%%

\section{联邦学习中的优化算法}
\addcontentsline{toe}{section}{{2.1\ \ Overview of Optimization Algorithms in Federated Learning}\numberline\,}

优化算法自从联邦学习这一概念诞生起,便一直是相关研究的中心问题。联邦学习开创性的文章\cite{mcmahan2017fed_avg}中,最重要也是最为人所熟知的便是优化算法\texttt{FederatedAveraging}(简记为\texttt{FedAvg})的提出。

待写。。。。

沿用式\eqref{eq:general-hfl}中的记号,本文考虑联邦学习中的优化问题,其最基本的格式如下
\begin{equation}
\label{eq:fl-basic-dist}
\begin{array}{cl}
\minimize\limits_{\theta \in \R^d} & f(\theta) = \expectation\limits_{k \sim {\mathcal{P}}} [f_k(\theta)], \\
\text{where} & f_k(\theta) = \expectation\limits_{(x, y) \sim \mathcal{D}_k} [\ell_k(\theta; x, y)],
\end{array}
\end{equation}
假设我们令$\mathcal{P} = \{1, 2, \ldots, K\},$ 则上述模型可以简记为
\begin{equation}
\label{eq:fl-basic}
\begin{array}{cl}
\minimize\limits_{\theta \in \R^d} & f(\theta) = \sum\limits_{k=1}^K w_k f_k(\theta).
\end{array}
\end{equation}
对于$f_k,$ 我们一般都假设它满足如下几条性质


\section{联邦学习中的原始对偶算法}
\addcontentsline{toe}{section}{{2.2\ \ Primal-Dual Algorithms in Federated Learning}\numberline\,}

待写。。。。

\section{联邦学习中的算子分裂算法}
\addcontentsline{toe}{section}{{2.3\ \ Operator Splitting Algorithms in Federated Learning}\numberline\,}

待写。。。。
