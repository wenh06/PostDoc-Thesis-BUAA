%\mbox{}\newpage
\chapter{\hspace{-1mm}\bf 联邦学习中的优化算法}
\label{chap2}
\addcontentsline{toe}{chapter}{{{\bf Chapter 2\ \
Optimization Algorithms in Federated Learning }}\numberline\,}
\markboth{第\,2\,章\ \
待写}{北京航空航天大学博士后研究工作报告}

%%%%%%%%%%%%%%%%%%%%%%%%%%%%%%%%%%%%%%%%%%%%%%%%%%%%%%%%%%%

\section{联邦学习中的优化算法}
\addcontentsline{toe}{section}{{2.1\ \ To write}\numberline\,}

优化算法自从联邦学习这一概念诞生起,便一直是其中心问题。联邦学习开创性的文章\cite{mcmahan2017fed_avg}中,最重要也是最为人所熟知的便是优化算法\texttt{FederatedAveraging}(简记为\texttt{FedAvg})的提出。

待写。。。。

一般来说,联邦学习中我们考虑的是如下的优化问题
\begin{equation}
\label{eq:fl-basic}
\begin{array}{cl}
\text{minimize} & f(x) = \expectation\limits_{i \sim {\mathcal{P}}} [f_i(x)] \\
\text{where} & f_i(x) = \expectation\limits_{z \sim \mathcal{D}_i} [\ell_i(x; z)]
\end{array}
\end{equation}
这里的$\mathcal{P}$为节点的分布,$\mathcal{D}_i$为节点$i$上 的数据分布,$\ell_i$为损失函数。

\section{联邦学习中的原始对偶算法}
\addcontentsline{toe}{section}{{2.2\ \ Primal-Dual Algorithms in Federated Learning}\numberline\,}

待写。。。。

\section{联邦学习中的算子分裂算法}
\addcontentsline{toe}{section}{{2.3\ \ Operator Splitting Algorithms in Federated Learning}\numberline\,}

待写。。。。
