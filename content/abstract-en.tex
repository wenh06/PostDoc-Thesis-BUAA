\chapter*{{Abstract}}%\vskip1cm
 \headheight=15.24pt%5mm
\markboth{Abstract} {Abstract}
 \headheight=15.24pt%5mm
\addcontentsline{toc}{chapter}{{{\bf Abstract}}\numberline ~~~}
\addcontentsline{toe}{chapter}{{{\bf Abstract in English}}\numberline ~}
\pagenumbering{Roman}\setcounter{page}{2}

% finished

Federated learning is an artificial intelligence technology that has emerged with the need for decentralization and privacy protection. It is a brand-new machine learning paradigm for scenarios where a central node (the server) coordinates multiple sub-nodes (the clients) into a collaborative modelling task. In this scenario, the participants can accomplish the task of collaborative modelling without exposing their private data. Currently, federated learning has become an active branch of machine learning and artificial intelligence.

This thesis mainly focuses on optimization algorithms in federated learning and their application in personalized federated learning. This thesis first introduces existing optimization algorithms in federated learning in detail, especially the decomposition algorithms. The basic idea of the decomposition algorithm is to decompose a large-scale problem into a series of small-scale problems and process each sub-problem separately so that the sub-problems in the iterative algorithms are easier to solve or easier to parallelize. Typical decomposition algorithms include operator splitting methods, alternating direction multiplier methods, etc. There are already a series of works showing the unique advantages of decomposition algorithms in the study of federated learning problems.

Subsequently, this thesis introduces algorithms for personalized federated learning and applies the decomposition algorithms to the research of personalized federated learning problems. Personalized federated learning is a method to deal with federated learning problems with complex data distribution. While training a public model collaboratively, it allows participants to obtain slightly different local models that retain their local characteristics, which improves the applicability of federated learning in related complex scenarios. There are still a lot of open problems in modelling and algorithm design in personalized federated learning. This thesis will conduct in-depth discussions on these problems and propose new algorithms and perspectives.

This thesis finishes with the introduction of a new simulation system designed for the verification of the federated learning algorithms, and numerical experiments using this simulation system to validate the algorithms proposed in this thesis in comparison to some existing algorithms.

\par
\bigskip

{\bf Key words:} Federated Learning, Collaborative Modeling, Decomposition Algorithms, Operator Splitting Methods, Model Personalization

%\nopagebreak[4]
\newpage
~~~\vspace{1em}
\thispagestyle{empty}

