\section{算子分裂法在个性化联邦学习中应用的初探}
\addcontentsline{toe}{section}{{\currentchapter .3\ \ A Short Survey on Operator Splitting Methods for Personalized Federated Learning}\numberline\,}
\label{sec:chap3-pfl-os}

% Chapter 4 is commented out and moved to this short section

% almost finished
% indexed

本报告已经在\S\ref{sec:chap2-operator-split}~中介绍了算子分裂法在联邦学习中的一些应用。算子分裂法以其对收敛结果正确性的保证\cite{pathak2020fedsplit}以及优美的收敛性理论\cite{ryu2022large}而在联邦学习的领域受到越来越多的关注。但是到本报告撰写的期间为止,还没有将算子分裂法应用到联邦学习个性化这一问题的研究出现。

首先值得考虑的问题是,个性化联邦学习的一般形式\eqref{eq:pfl-general}中的正则项$\mathcal{R}.$ 已有的大部分关于个性化联邦学习的研究\cite{hanzely2020federated, li_2021_ditto, t2020pfedme}都采用的是二次罚项。是否有更合适的正则项,来达到个性化的目的,是值得研究的问题。

其次,考虑如下建模方式的个性化联邦学习问题
\begin{equation}
\label{eq:customized-drsm}
\begin{array}{cl}
 \minimize & \sum\limits_{k=1}^K \left\{ f_k(\theta_k) + g(\delta_k) \right\}, \\
 \text{s.t.} & \omega_i = \theta_k - \delta_k, \\
 & \omega_1 = \cdots = \omega_K.
\end{array}
\end{equation}
其中$g$是某种形式的正则项。以上问题的格式在优化目标内将目标函数$f$与正则项$g$进行了变量分离,那么可以考虑设计定制化的DR分裂方法 (Customized Douglas-Rachford Splitting Method)\cite{Han_2013_CDRSM}\index{定制化的DR分裂方法, Customized Douglas-Rachford Splitting Method)}进行优化求解。如何设计适合联邦学习场景的定制DR分裂方法,是一件有挑战性的工作。

同时,我们也注意到了,参考文献\parencite{tran2021feddr}~实质上把联邦学习的优化问题建模为了一个三项求和形式的问题 (见式\eqref{eq:feddr-pen})。为了将DRSM用到这个问题的求解上,参考文献\parencite{tran2021feddr}~将后两项求和视为一项。事实上,对于这类问题的处理,可以直接利用Davis–Yin分裂方法(Davis–Yin Splitting Method, DYSM)\cite{ryu2022large, Davis_2017_DYS, Liu_2019_DYS}\index{Davis–Yin分裂方法, Davis–Yin Splitting Method, DYSM},相关算法的研究工作在联邦学习领域将会是开创性的。
