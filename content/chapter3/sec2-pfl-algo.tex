\section{个性化联邦学习中的典型算法}
\addcontentsline{toe}{section}{{3.2\ \ Existing Algorithms for Personalized Federated Learning}\numberline\,}
\label{sec:chap3-pfl-algo}

% NOT finished
% NOT indexed

本节主要介绍已有的一些基于正则化目标函数的个性化联邦学习算法,此类算法的基本形式见上一节\S~\ref{sec:chap3-pfl}~的式~\eqref{eq:pfl-general}。这一类算法形式最简单的是文献\cite{hanzely2020federated}所引入的无环随机梯度下降 (Loopless Local SGD, L2SGD)\index{无环随机梯度下降, Loopless Local SGD, L2SGD} 算法。其考虑的优化问题格式为
\begin{equation}
\label{eq:l2sgd}
\begin{array}{cl}
\minimize & F(\Theta) = \frac{1}{K} \sum\limits_{k=1}^K f_i(\theta_i) + \lambda \varphi(\Theta) \\
\text{where} & \varphi(\Theta) = \frac{1}{2K} \sum\limits_{k=1}^K \left\lVert \theta_k - \bar{\theta} \right\rVert^2 = \frac{1}{2K} \sum\limits_{k=1}^K \left\lVert \theta_k - \frac{1}{K} \sum\limits_{j=1}^K \theta_j \right\rVert^2 \\
& \Theta = \col(\theta_1, \ldots, \theta_K).
\end{array}
\end{equation}
其中$\lambda \geqslant 0$是罚参数。当$0 < \lambda < \infty,$ 求解上述优化问题得到的模型$\bar{\theta}, \theta_1, \ldots, \theta_K$被称作是混合模型 (Mixed Models)。很容易看出,上述问题可以等价地转化为联邦临近算法\texttt{FedProx}所考察的问题~\eqref{eq:fedprox-whole}。文献\cite{hanzely2020federated}对子节点上的目标函数做了进一步的假设,即假设$f_k$有有限和的结构
\begin{equation*}
f_k = \sum\limits_{j=1}^m f_{k,j},
\end{equation*}
并基于随机块坐标下降方法 (Randomized Block Coordinate Descent, RBCD)\index{随机块坐标下降, Randomized Block Coordinate Descent, RBCD},为问题~\eqref{eq:l2sgd}~的求解设计了所谓的无环随机梯度下降法。具体来说,待写。。。。

\begin{algorithm}[ht]
% \SetAlgoNoLine
% \DontPrintSemicolon
\SetKwInOut{Input}{Input}
\Input{xx}
{\bfseries Initiation:}\;
\caption{算法\texttt{L2SGD}\cite{hanzely2020federated}的伪代码}
\label{algo:l2sgd}
\end{algorithm}


\begin{algorithm}[ht]
% \SetAlgoNoLine
% \DontPrintSemicolon
{\bfseries Input:} $T, R, S, \lambda, \eta, \beta, \theta^{(0)} \in \R^d$\;
\For{each round $t = 0, \cdots, T-1$}{
    Server sends $\theta^{(t)}$ to all clients\;
    \For{each client $k = 1, \cdots, K$ in parallel}{
        $\theta_{k}^{(t, 0)} = \theta^{(t)}$\;
        \For{$r = 0,\cdots, R-1$}{
            Sample a mini-batch $b_r$\;
            find an approximate $\theta_k(\theta_{k}^{(t, r)})$ to the problem $\text{min} \{ \ell_k(\theta_k; b_r) + \frac{\lambda}{2} \left\lVert \theta_k - \theta_{k}^{(t, r)} \right\rVert^2 \}$\;
            Local update $\theta_{k}^{(t, r+1)} = \theta_{k}^{(t, r)} - \eta \lambda \left( \theta_{k}^{(t, r)} - \theta_k(\theta_{k}^{(t, r)}) \right)$\;
        }
        Server uniformly samples a subset of clients $\mathcal{S}^{(t)}$,\;
        each client in $\mathcal{S}^{(t)}$ sends the local $\theta_{k}^{(t, R)}$ to the server\;
    }
    Sever update $\theta^{(t+1)} = (1-\beta)\theta^{(t)} + \beta \sum \frac{\theta_{k}^{(t, R)}}{\# \mathcal{S}^t}$
}
\caption{\texttt{pFedMe}\cite{t2020pfedme}算法伪代码}
\label{algo:pfedme}
\end{algorithm}


文献\parencite{li_2021_ditto}进一步发展了\texttt{FedProx}\cite{sahu2018fedprox}添加临近项的思想,将子节点的优化问题~\eqref{eq:fedprox}~改进为一个双层优化问题 (Bi-level Optimization Problem)\index{双层优化问题, Bi-level Optimization Problem}
\begin{equation}
\label{eq:ditto-local}
\begin{array}{cl}
\minimize & h_k(\theta_k, \omega^*) := f_k(\theta_k) + \frac{\mu}{2} \lVert \theta_k - \omega^* \rVert^2, \\
\text{subject to} & \omega^* \in \argmin_{\omega} G(f_1(\omega), \cdots, f_K(\omega)),
\end{array}
\end{equation}
其中$G$是中心节点上执行子节点模型参数聚合的函数,例如\texttt{FedAvg}算法中采用的求均值函数。

待写。。。。

\begin{algorithm}[ht]
% \SetAlgoNoLine
% \DontPrintSemicolon
\SetKwInOut{Input}{Input}
\Input{penalty coeffecient $\mu,$ learning rate $\eta,$ methods $\operatorname{\mathbf{UpdateGlobal}}, \operatorname{\mathbf{Aggregate}}$}
{\bfseries Initiation:}\;
\Indp
    {\bfseries Init server:} global model parameters $\theta^{(0)} \in \R^d$\;
    {\bfseries Init clients:} local model parameters $\omega_k^{(0)} \in \R^d, ~ \forall k \in [K]$\;
\Indm
\For{each round $t = 0, 1, \cdots, T-1$}{
    $\mathcal{S}^{(t)} \gets$ (random set of clients) $\subseteq [K]$\;
    broadcast $\theta^{(t)}$ to clients $k \in \mathcal{S}^{(t)}$\;
    \For{each client $k \in \mathcal{S}^{(t)}$ {\bfseries in parallel}}{
        \tcc{solve the local sub-problem of $G(\dot)$ inexactly starting from $\theta^{(t)}$ to obtain $\theta_k^{(t)}$}
        $\theta_k^{(t)} \gets \operatorname{\mathbf{UpdateGlobal}}(\theta^{(t)}, f_k)$\;
        \tcc{update the personalized model via solving \eqref{eq:ditto-local}}
        $\omega_k^{(t+1)} \gets \omega_k^{(t)} - \eta \left( \nabla f_k(\omega_k^{(t)}) + \mu (\omega_k^{(t)} - \theta^{(t)}) \right)$\;
        send $\Delta_k^{(t)} \gets \theta_k^{(t)} - \theta^{(t)}$ to server\;
    }
    {\bfseries Server Update:}\;
    \Indp
    $\theta^{(t+1)} \gets \operatorname{\mathbf{Aggregate}} (\theta^{(t)}, \{ \Delta_k^{(t)} \}_{k \in \mathcal{S}^{(t)}})$\;
    \Indm
}
\caption{算法\texttt{Ditto}\cite{li_2021_ditto}的伪代码}
\label{algo:ditto}
\end{algorithm}


\begin{algorithm}[!htb]
% \SetAlgoNoLine
% \DontPrintSemicolon
\SetKwInOut{Input}{Input}
\Input{mixture weights $\alpha_1, \ldots, \alpha_K,$ synchronization gap $\tau$}
{\bfseries Initiation:}\;
\Indp
    {\bfseries Init server:} global model parameters $\theta^{(0)} \in \R^d,$ random set of clients $S^{(t)} \subseteq [K]$\;
    {\bfseries Init clients:} local model parameters $\omega_k^{(0)} \in \R^d, ~ \theta_k^{(0)} \gets \theta^{(0)}, ~ \forall k \in [K]$\;
    {\bfseries Constants:} condition number $\kappa \gets \frac{L}{\mu},$ $a \gets \max\{128\kappa, \tau\}$\;
\Indm
\For{each round $t = 0, 1, \cdots, T-1$}{
    \For{each client $k \in \mathcal{S}^{(t)}$ {\bfseries in parallel}}{
        $\bar{\omega}_k^{(t)} \gets \alpha_k \omega_k^{(t)} + (1-\alpha_k) \theta_k^{(t)}$ \tcc*[h]{mixture model}\;
        $\eta^{(t)} \gets \frac{16}{\mu(t+a)}$ \tcc*[h]{decay learning rate}\;
        $\theta_k^{(t+1)} \gets \theta_k^{(t)} - \eta^{(t)} \nabla f_k(\theta_k^{(t)})$ \tcc*[h]{inner problem (global model) update}\;
        $\omega_k^{(t+1)} \gets \omega_k^{(t)} - \eta^{(t)} \nabla f_k(\bar{\omega}_k^{(t)})$ \tcc*[h]{outer problem (personalized model) update}\;
        \tcc{Optional: adaptive mixture weights update}
        \tcc{$\alpha_k \gets \alpha_k - \eta^{(t)} \nabla_{\alpha_k} f_k(\bar{\omega}_k^{(t)}) = \alpha_k - \eta^{(t)} \left\langle \omega_k^{(t)} - \theta_k^{(t)}, \nabla f_k(\bar{\omega}_k^{(t)}) \right\rangle$}
    }
    \uIf{$t$ not divides synchronization gap $\tau$}{
        {\bfseries Server updates}: $\mathcal{S}^{(t+1)} \gets \mathcal{S}^{(t)}$\;
    }
    \Else{
        each client $k \in \mathcal{S}^{(t)}$ sends $\theta_k^{(t+1)}$ to server\;
        {\bfseries Server updates}:\;
        \Indp
        $\theta^{(t+1)} \gets \frac{1}{\# \mathcal{S}^{(t)}} \theta_k^{(t+1)}$\;
        $\mathcal{S}^{(t+1)} \gets$ (random set of clients) $\subseteq [K]$\;
        broadcast $\theta^{(t+1)}$ to clients $k \in S^{(t+1)}:$ $\theta_k^{(t+1)} \gets \theta^{(t+1)}$\;
        \Indm
    }
final personalized model: $\hat{\omega}_k \gets \frac{1}{S_T}\sum\limits_{t=1}^{T} p_t \left( \alpha_k \omega_k^{(t)} + (1-\alpha_k)\frac{1}{\#\mathcal{S}^{(t-1)}}\sum\limits_{k\in\mathcal{S}^{(t-1)}}\theta_k^{(t)} \right)$\;
final global model: $\hat{\theta} \gets \frac{1}{S_T}\sum\limits_{t=1}^{T} \frac{p_t}{\#\mathcal{S}^{(t-1)}} \sum\limits_{k\in\mathcal{S}^{(t-1)}}\theta_k^{(t)}$\;
where $p_t = (t+a)^2, S_T = \sum\limits_{t=1}^{T}p_t$\;
}
\caption{算法\texttt{APFL}\cite{deng2020_apfl}的伪代码}
\label{algo:apfl}
\end{algorithm}


\begin{algorithm}[ht]
% \SetAlgoNoLine
% \DontPrintSemicolon
\SetKwInOut{Input}{Input}
\Input{penalty coeffecient $\mu$}
{\bfseries Initiation:}\;
\Indp
    {\bfseries Init server:} global model parameters $\theta^{(0)} \in \R^d,$ $h = 0 \in \R^d$\;
    {\bfseries Init clients:} local gradient $\mathfrak{g}_k^{(0)} \gets 0 \in \R^d, ~ \forall k \in [K]$\;
\Indm
\For{each round $t = 0, 1, \cdots, T-1$}{
    $S^{(t)} \gets$ (random set of clients) $\subseteq [K]$\;
    broadcast $\theta^{(t)}$ to clients $k \in S^{(t)}$\;
    \For{each client $k \in \mathcal{S}^{(t)}$ {\bfseries in parallel}}{
        $\theta_k^{(t+1)} \gets \argmin\limits_{\theta_k} \left\{ f_k(\theta_k) - \langle \mathfrak{g}_k^{(t)}, \theta_k \rangle + \frac{\mu}{2} \lVert \theta_k - \theta^{(t)} \rVert^2 \right\}$ \;
        $\mathfrak{g}_k^{(t+1)} \gets \mathfrak{g}_k^{(t)} - \mu (\theta_k^{(t+1)} - \theta^{(t)})$\;
        send $\theta_k^{(t+1)}$ to server\;
    }
    {\bfseries Server Update:}\;
    \Indp
    $h^{(t+1)} \gets h^{(t)} - \frac{1}{\mu} \left(\sum\limits_{k \in S^{(t)}} \theta_k^{(t+1)} - \theta^{(t)} \right)$\;
    $\theta^{(t+1)} \gets \left( \frac{1}{\# S^{(t)}}\sum\limits_{k \in S^{(t)}} \theta_k^{(t+1)} \right) - \frac{1}{\mu} h^{(t+1)}$\;
}
\caption{算法\texttt{FedDyn}\cite{acar2021feddyn}的伪代码}
\label{algo:feddyn}
\end{algorithm}


\begin{equation}
\label{eq:pfedmac}
\begin{array}{cl}
\minimize & F(\theta) = \frac{1}{K} \sum\limits_{k=1}^K F_i(\theta), \\
\text{where} & F_k(\theta) := \min\limits_{\theta_k} \left\{ f_k(\theta_k) - \mu \langle \theta_k, \theta \rangle + \frac{\mu}{2} \lVert \theta \rVert^2_2 \right\}
\end{array}
\end{equation}


\begin{equation}
\label{eq:fedu}
\minimize \quad \sum\limits_{k=1}^K f_k(x_k) + \dfrac{\lambda}{2} \sum\limits_{k=1}^K \sum\limits_{j\in\mathcal{N}_k} \lVert \theta_k - \theta_j \rVert^2
\end{equation}

\begin{equation}
\label{eq:pfedmac-sparse}
\begin{array}{cl}
\minimize & F(\theta) = \frac{1}{K} \sum\limits_{k=1}^K F_i(\theta), \\
\text{where} & F_k(\theta) := \min\limits_{\theta_k} \left\{ f_k(\theta_k) - \mu \langle \theta_k, \theta \rangle + \frac{\mu}{2} \lVert \theta \rVert^2_2 + \lVert \theta_k \rVert_1 \right\}
\end{array}
\end{equation}

\begin{algorithm}[ht]
% \SetAlgoNoLine
% \DontPrintSemicolon
\SetKwInOut{Input}{Input}
\Input{learning rate $\eta$, penalty coefficient $\lambda$, $\beta$}
% {\bfseries Server executes:}\;
% \Indp
{\bfseries Initiation:} global (server) model parameters $\theta^{(0)} \in \R^d$\;
\For{each round $t = 0, 1, \cdots, T-1$}{
    $\mathcal{S}^{(t)} \gets$ (random set of clients) $\subseteq [K]$\;
    broadcast $\theta^{(t)}$ to clients $k \in \mathcal{S}^{(t)}$\;
    \For{each client $k \in \mathcal{S}^{(t)}$ {\bfseries in parallel}}{
        $\theta_k^{(t)} \gets$ {\bfseries ClientUpdate}$(k, \theta^{(t)})$\;
        send $\theta_k^{(t)}$ to server\;
    }
    {\bfseries Server Update:}\;
    \Indp
    $\theta^{(t+1)} \gets (1 - \beta) \theta^{(t)} + \frac{\beta}{\lvert \mathcal{S}^{(t)} \rvert} \sum\limits_{k\in \mathcal{S}^{(t)}} \theta_k^{(t)}$\;
    \Indm
}
% \Indm
\vspace{0.2em}
{\bfseries ClientUpdate}$(k, \theta)$: \tcc*[h]{on client $k$}\;
\Indp
$\omega_k^{(t,0)} = \theta_k^{(t,0)} = \theta^{(t)}$\;
\For{local step $r = 0, 1, \cdots, R-1$}{
    $\mathcal{D}_{k, r} \gets$ (sample a mini-batch data)\;
    $\omega_k^{(t,r)} \gets \argmin_{\omega_k} \left\{ \ell_k(\omega_k; \mathcal{D}_{k, r}) - \lambda \langle \omega_k, \theta_k^{(t,r)} \rangle \right\}$\;
    $\theta_k^{(t,r+1)} \gets \theta_k^{(t,r)} - \eta\lambda \left( \theta_k^{(t,r)} - \omega_k^{(t,r)} \right)$\;
}
\Return{$\theta_k^{(t,R)}$}
\caption{算法\texttt{pFedMac}\cite{li2021pfedmac}的伪代码}
\label{algo:pfedmac}
\end{algorithm}


待写。。。。
