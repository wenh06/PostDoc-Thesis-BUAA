\section{个性化联邦学习中的典型算法}
\addcontentsline{toe}{section}{{3.2\ \ Existing Algorithms for Personalized Federated Learning}\numberline\,}
\label{sec:chap3-pfl-algo}

% NOT finished
% NOT indexed

待写。。。。

\begin{algorithm}[ht]
% \SetAlgoNoLine
% \DontPrintSemicolon
\SetKwInOut{Input}{Input}
\Input{xx}
{\bfseries Initiation:}\;
\caption{算法\texttt{L2SGD}\cite{hanzely2020federated}的伪代码}
\label{algo:l2sgd}
\end{algorithm}


\begin{algorithm}[ht]
% \SetAlgoNoLine
% \DontPrintSemicolon
{\bfseries Input:} $T, R, S, \lambda, \eta, \beta, \theta^{(0)} \in \R^d$\;
\For{each round $t = 0, \cdots, T-1$}{
    Server sends $\theta^{(t)}$ to all clients\;
    \For{each client $k = 1, \cdots, K$ in parallel}{
        $\theta_{k}^{(t, 0)} = \theta^{(t)}$\;
        \For{$r = 0,\cdots, R-1$}{
            Sample a mini-batch $b_r$\;
            find an approximate $\theta_k(\theta_{k}^{(t, r)})$ to the problem $\text{min} \{ \ell_k(\theta_k; b_r) + \frac{\lambda}{2} \left\lVert \theta_k - \theta_{k}^{(t, r)} \right\rVert^2 \}$\;
            Local update $\theta_{k}^{(t, r+1)} = \theta_{k}^{(t, r)} - \eta \lambda \left( \theta_{k}^{(t, r)} - \theta_k(\theta_{k}^{(t, r)}) \right)$\;
        }
        Server uniformly samples a subset of clients $\mathcal{S}^{(t)}$,\;
        each client in $\mathcal{S}^{(t)}$ sends the local $\theta_{k}^{(t, R)}$ to the server\;
    }
    Sever update $\theta^{(t+1)} = (1-\beta)\theta^{(t)} + \beta \sum \frac{\theta_{k}^{(t, R)}}{\# \mathcal{S}^t}$
}
\caption{\texttt{pFedMe}\cite{t2020pfedme}算法伪代码}
\label{algo:pfedme}
\end{algorithm}


文献\parencite{li_2021_ditto}进一步发展了\texttt{FedProx}\cite{sahu2018fedprox}添加临近项的思想,将子节点的优化问题~\eqref{eq:fedprox}~改进为一个双层优化问题 (Bi-level Optimization Problem)\index{双层优化问题, Bi-level Optimization Problem}
\begin{equation}
\label{eq:ditto-local}
\begin{array}{cl}
\minimize & h_k(\theta_k, \omega^*) := f_k(\theta_k) + \frac{\mu}{2} \lVert \theta_k - \omega^* \rVert^2, \\
\text{subject to} & \omega^* \in \argmin_{\omega} G(f_1(\omega), \cdots, f_K(\omega)),
\end{array}
\end{equation}
其中$G$是中心节点上执行子节点模型参数聚合的函数,例如\texttt{FedAvg}算法中采用的求均值函数。

待写。。。。

\begin{algorithm}[ht]
% \SetAlgoNoLine
% \DontPrintSemicolon
\SetKwInOut{Input}{Input}
\Input{penalty coeffecient $\mu,$ learning rate $\eta,$ methods $\operatorname{\mathbf{UpdateGlobal}}, \operatorname{\mathbf{Aggregate}}$}
{\bfseries Initiation:}\;
\Indp
    {\bfseries Init server:} global model parameters $\theta^{(0)} \in \R^d$\;
    {\bfseries Init clients:} local model parameters $\omega_k^{(0)} \in \R^d, ~ \forall k \in [K]$\;
\Indm
\For{each round $t = 0, 1, \cdots, T-1$}{
    $\mathcal{S}^{(t)} \gets$ (random set of clients) $\subseteq [K]$\;
    broadcast $\theta^{(t)}$ to clients $k \in \mathcal{S}^{(t)}$\;
    \For{each client $k \in \mathcal{S}^{(t)}$ {\bfseries in parallel}}{
        \tcc{solve the local sub-problem of $G(\dot)$ inexactly starting from $\theta^{(t)}$ to obtain $\theta_k^{(t)}$}
        $\theta_k^{(t)} \gets \operatorname{\mathbf{UpdateGlobal}}(\theta^{(t)}, f_k)$\;
        \tcc{update the personalized model via solving \eqref{eq:ditto-local}}
        $\omega_k^{(t+1)} \gets \omega_k^{(t)} - \eta \left( \nabla f_k(\omega_k^{(t)}) + \mu (\omega_k^{(t)} - \theta^{(t)}) \right)$\;
        send $\Delta_k^{(t)} \gets \theta_k^{(t)} - \theta^{(t)}$ to server\;
    }
    {\bfseries Server Update:}\;
    \Indp
    $\theta^{(t+1)} \gets \operatorname{\mathbf{Aggregate}} (\theta^{(t)}, \{ \Delta_k^{(t)} \}_{k \in \mathcal{S}^{(t)}})$\;
    \Indm
}
\caption{算法\texttt{Ditto}\cite{li_2021_ditto}的伪代码}
\label{algo:ditto}
\end{algorithm}


\begin{algorithm}[ht]
% \SetAlgoNoLine
% \DontPrintSemicolon
\SetKwInOut{Input}{Input}
\Input{penalty coeffecient $\mu$}
{\bfseries Initiation:}\;
\Indp
    {\bfseries Init server:} global model parameters $\theta^{(0)} \in \R^d,$ $h = 0 \in \R^d$\;
    {\bfseries Init clients:} local gradient $\mathfrak{g}_k^{(0)} \gets 0 \in \R^d, ~ \forall k \in [K]$\;
\Indm
\For{each round $t = 0, 1, \cdots, T-1$}{
    $S^{(t)} \gets$ (random set of clients) $\subseteq [K]$\;
    broadcast $\theta^{(t)}$ to clients $k \in S^{(t)}$\;
    \For{each client $k \in \mathcal{S}^{(t)}$ {\bfseries in parallel}}{
        $\theta_k^{(t+1)} \gets \argmin\limits_{\theta_k} \left\{ f_k(\theta_k) - \langle \mathfrak{g}_k^{(t)}, \theta_k \rangle + \frac{\mu}{2} \lVert \theta_k - \theta^{(t)} \rVert^2 \right\}$ \;
        $\mathfrak{g}_k^{(t+1)} \gets \mathfrak{g}_k^{(t)} - \mu (\theta_k^{(t+1)} - \theta^{(t)})$\;
        send $\theta_k^{(t+1)}$ to server\;
    }
    {\bfseries Server Update:}\;
    \Indp
    $h^{(t+1)} \gets h^{(t)} - \frac{1}{\mu} \left(\sum\limits_{k \in S^{(t)}} \theta_k^{(t+1)} - \theta^{(t)} \right)$\;
    $\theta^{(t+1)} \gets \left( \frac{1}{\# S^{(t)}}\sum\limits_{k \in S^{(t)}} \theta_k^{(t+1)} \right) - \frac{1}{\mu} h^{(t+1)}$\;
}
\caption{算法\texttt{FedDyn}\cite{acar2021feddyn}的伪代码}
\label{algo:feddyn}
\end{algorithm}


\begin{equation}
\label{eq:pfedmac}
\begin{array}{cl}
\minimize & F(\theta) = \frac{1}{K} \sum\limits_{k=1}^K F_i(\theta), \\
\text{where} & F_k(\theta) := \min\limits_{\theta_k} \left\{ f_k(\theta_k) - \mu \langle \theta_k, \theta \rangle + \frac{\mu}{2} \lVert \theta \rVert^2_2 \right\}
\end{array}
\end{equation}


\begin{equation}
\label{eq:pfedmac-sparse}
\begin{array}{cl}
\minimize & F(\theta) = \frac{1}{K} \sum\limits_{k=1}^K F_i(\theta), \\
\text{where} & F_k(\theta) := \min\limits_{\theta_k} \left\{ f_k(\theta_k) - \mu \langle \theta_k, \theta \rangle + \frac{\mu}{2} \lVert \theta \rVert^2_2 + \lVert \theta_k \rVert_1 \right\}
\end{array}
\end{equation}

\begin{algorithm}[ht]
% \SetAlgoNoLine
% \DontPrintSemicolon
\SetKwInOut{Input}{Input}
\Input{learning rate $\eta$, penalty coefficient $\lambda$, $\beta$}
% {\bfseries Server executes:}\;
% \Indp
{\bfseries Initiation:} global (server) model parameters $\theta^{(0)} \in \R^d$\;
\For{each round $t = 0, 1, \cdots, T-1$}{
    $\mathcal{S}^{(t)} \gets$ (random set of clients) $\subseteq [K]$\;
    broadcast $\theta^{(t)}$ to clients $k \in \mathcal{S}^{(t)}$\;
    \For{each client $k \in \mathcal{S}^{(t)}$ {\bfseries in parallel}}{
        $\theta_k^{(t)} \gets$ {\bfseries ClientUpdate}$(k, \theta^{(t)})$\;
        send $\theta_k^{(t)}$ to server\;
    }
    {\bfseries Server Update:}\;
    \Indp
    $\theta^{(t+1)} \gets (1 - \beta) \theta^{(t)} + \frac{\beta}{\lvert \mathcal{S}^{(t)} \rvert} \sum\limits_{k\in \mathcal{S}^{(t)}} \theta_k^{(t)}$\;
    \Indm
}
% \Indm
\vspace{0.2em}
{\bfseries ClientUpdate}$(k, \theta)$: \tcc*[h]{on client $k$}\;
\Indp
$\omega_k^{(t,0)} = \theta_k^{(t,0)} = \theta^{(t)}$\;
\For{local step $r = 0, 1, \cdots, R-1$}{
    $\mathcal{D}_{k, r} \gets$ (sample a mini-batch data)\;
    $\omega_k^{(t,r)} \gets \argmin_{\omega_k} \left\{ \ell_k(\omega_k; \mathcal{D}_{k, r}) - \lambda \langle \omega_k, \theta_k^{(t,r)} \rangle \right\}$\;
    $\theta_k^{(t,r+1)} \gets \theta_k^{(t,r)} - \eta\lambda \left( \theta_k^{(t,r)} - \omega_k^{(t,r)} \right)$\;
}
\Return{$\theta_k^{(t,R)}$}
\caption{算法\texttt{pFedMac}\cite{li2021pfedmac}的伪代码}
\label{algo:pfedmac}
\end{algorithm}


待写。。。。
