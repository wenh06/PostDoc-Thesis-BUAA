\section{联邦学习中的临近点算法}
\addcontentsline{toe}{section}{{2.2\ \ Proximal Point Algorithms in Federated Learning}\numberline\,}
\label{sec:chap2-ppa}

基于减小数据非独立同分布情形下,每一轮的子节点更新对整体模型的影响,文献\parencite{sahu2018fedprox}首先提在子节点上的局部模型中增加邻近项的\texttt{FedProx}算法,达到了算法收敛更快、更加稳定的目的。具体来说,在算法迭代的第$t+1$轮,每个子节点优化的目标函数从$f_k(\theta_k)$变为了如下的带临近项的函数
\begin{equation}
\label{eq:fedprox}
f_k(\theta_k) + \frac{\mu}{2} \lVert \theta_k - \theta^{(t)} \rVert^2
\end{equation}
值得注意的是,上式临近项的中心为$\theta^{(t)},$ 即上一轮迭代得到的全局模型。关于临近项中心的其他选取,文献\parencite{hanzely2020federated,li_2021_ditto}进行了研究,我们随后进行详细介绍。

待写。。。。


% fedprox
bounded dissimilarity:
for some $\epsilon > 0$, $\exists B_{\epsilon}$ s.t. $\forall w \in \{ w ~|~ \lVert \nabla f(w) > \epsilon \rVert \}$, $B(w) \leqslant B$.

一些关于FedProx文章中的Theorem 4的观察:

在$\lVert \nabla f \rVert$的零点附近,如果这个零点没有被$\mathbb{E}_k[\lVert \nabla F_k \rVert]$消除掉的话,$B$会急速趋向于无穷,导致在$\lVert \nabla f \rVert$的零点附近,$\rho > 0$的假设不再成立,那么定理中的不等式就变得无意义了。当子节点之间的数据分布完全一致的时候(理想情况下),$B$恒为1,就不会有这个问题。这也是FedSplit\cite{pathak2020fedsplit}文章里提到的。
