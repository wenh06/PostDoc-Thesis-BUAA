\section{联邦学习中的算子分裂算法}
\addcontentsline{toe}{section}{{2.3\ \ Operator Splitting Algorithms in Federated Learning}\numberline\,}
\label{sec:chap2-operator-split}

我们已经讨论过的通用加速技巧\cite{reddi2020fed_opt},方差缩减技术\cite{karimireddy2020scaffold},以及临近点方法\cite{sahu2018fedprox},解决的都是联邦学习在各种各样场景下的收敛性、收敛率相关的问题,而算法收敛结果的正确性这一问题始终没有得到重视。我们来举一个非常简单的例子来说明这个问题。

\begin{example}
\label{eg:correctness}
假设我们要拟合线性模型$y = mx$, $m$为需要拟合的变量,使用均方误差(Mean Squared Error, MSE)做损失函数 (目标函数)。我们有两个参与方,参与方1的拥有的训练数据为$\{ (0, 2), (1, 2) \},$ 相应的目标函数为
\begin{equation*}
f_1(m_1) = 4 + (m_1 - 2)^2;
\end{equation*}
参与方2的拥有的训练数据为$\{ (2, 0), (2, 1) \},$ 相应的目标函数为
\begin{equation*}
f_2(m_2) = 4m_2^2 + (2m_2 - 1)^2.
\end{equation*}
那么参与方1单独拟合结果为$m_1 = 2$, 参与方2单独拟合结果为$m_2 = \frac{1}{4}$, 采用联邦平均算法\texttt{FedAvg}的结果是$m = \frac{9}{8}.$ 而采用联邦临近点算法\texttt{FedProx}的结果(计算依据见定理\ref{thm:fedsplit-correctness})是
\begin{equation}
\label{eq:eg-correctness-fedprox}
m = \frac{36 + 4\mu}{32 + 9\mu},
\end{equation}
其中$\mu$为临近项相关的系数 (见式~\eqref{eq:fedprox})。

数据都拿一起拟合结果为$k = \frac{4}{9}.$ 注意,这里一起拟合的结果并不等于单独拟合的均值 (\texttt{FedAvg}的结果),原因就在于loss function里的k是2次的,不是线性的。
\end{example}

文献\cite{pathak2020fedsplit}首先注意到这个问题,并具体分析了联邦平均算法\texttt{FedAvg}与联邦临近点算法\texttt{FedProx}的理论收敛结果,如下

\begin{theorem}
\label{thm:fedsplit-correctness}
假设我们有$K$个节点参与联邦学习,每一个节点$k \in [K]$的目标函数为$f_k(\theta_k),$ 这些函数都是有限凸函数,而且都是$L$-光滑的(定义见式~\eqref{eq:l-smooth}),那么
\begin{itemize}
\item[(1)] 若采用联邦平均算法\texttt{FedAvg} \ref{algo:fedavg},且假设子节点上的参数更新方式为全量的梯度下降。我们把这种简化的算法记为\texttt{FedGD},记梯度映射为
\begin{equation}
\label{eq:grad-mapping}
\mathcal{G}_k (\theta_k) := \theta_k - \eta \nabla f_k (\theta_k),
\end{equation}
并记$\mathcal{G}^i_k (\theta_k) := \underbrace{G_k\circ\cdots\circ G_k}_{i-\text{次复合}} (\theta_k).$ 若算法 \ref{algo:fedavg}~生成的全局模型参数的序列$\{ \theta^{(t)} \}_{t=1}^{\infty}$收敛,那么所有子节点$k \in [K]$模型参数的序列$\{ \theta_k^{(t)} \}_{t=1}^{\infty}$有公共的极限$\theta^*,$ 且$\theta^*$满足下列不动点条件
\begin{equation}
\label{eq:fedgd-fixed-pt}
\sum\limits_{i=1}^R \sum\limits_{k=1}^K \nabla f_k(\mathcal{G}_k^{i-1}(\theta^*)) = 0.
\end{equation}
\item[(2)] 若采用联邦临近点算法\texttt{FedProx} \ref{algo:fedprox},记Moreau包络映射~\eqref{eq:moreau_env}~为
\begin{equation}
\label{eq:moreau-mapping}
\mathcal{M}_{\mu, f_k} (\theta) := \inf\limits_{\theta_k} \left\{ f_k(\theta_k) + \frac{\mu}{2} \lVert \theta - \theta_k \rVert^2 \right\}.
\end{equation}
若算法 \ref{algo:fedprox}~生成的全局模型参数的序列$\{ \theta^{(t)} \}_{t=1}^{\infty}$收敛,那么所有子节点$k \in [K]$模型参数的序列$\{ \theta_k^{(t)} \}_{t=1}^{\infty}$有公共的极限$\theta^*,$ 且$\theta^*$满足下列不动点条件
\begin{equation}
\label{eq:fedprox-fixed-pt}
\sum\limits_{k=1}^K \nabla \mathcal{M}_{\mu, f_k} (\theta^*) = 0.
\end{equation}
\end{itemize}
\end{theorem}

\begin{proof}
\begin{itemize}
\item[(1)] 记$\Theta^{(t)} = (\theta_1^{(t)}, \cdots, \theta_K^{(t)}).$ 由于算法 \ref{algo:fedavg}~生成的全局模型参数的序列$\{ \theta^{(t)} \}_{t=1}^{\infty}$收敛,也就是说约束优化问题~\eqref{eq:fedavg-constraint}~有解$\Theta^* = (\theta_1^*, \cdots, \theta_K^*).$ 这个解要满足问题~\eqref{eq:fedavg-constraint}~的约束条件,即满足
\begin{equation*}
\theta_1^* = \cdots = \theta_K^*,
\end{equation*}
那么我们就知道了所有子节点$k \in [K]$模型参数的序列$\{ \theta_k^{(t)} \}_{t=1}^{\infty}$有公共的极限$\theta^* = \theta_1^* = \cdots = \theta_K^*.$ 全局模型参数序列
\begin{equation*}
\theta^{(t)} = \frac{1}{K} (\theta_1^{(t)} + \cdots + \theta_K^{(t)}), ~~ t = 1, \ldots
\end{equation*}
也收敛于$\theta^*,$ 从而有
\begin{equation*}
\theta^* = \texttt{FedGD} (\theta^*) := \frac{1}{K} ( \mathcal{G}^R_1 (\theta^*) + \cdots + \mathcal{G}^R_K (\theta^*) )
\end{equation*}
即
\begin{equation*}
0 = \theta^* - \frac{1}{K} \sum\limits_{k=1}^K \mathcal{G}^R_k (\theta^*)
\end{equation*}
我们把梯度映射~\eqref{eq:grad-mapping}~的定义反复代入上式中,即有
\begin{align*}
0 = & \frac{1}{K}\sum_{k=1}^K \mathcal{G}^R_k(\theta^{*}) - \theta^{*} = \frac{1}{K} \sum_{k=1}^K \mathcal{G}_k ( \mathcal{G}^{R-1}_k (\theta^{*}) ) - \theta^{*} \\
= & \frac{1}{K}\sum_{k=1}^K \left( \mathcal{G}^{R-1}_k(\theta^{*}) - \eta\nabla f_k (\mathcal{G}^{R-1}_k(\theta^{*})) \right) - \theta^{*} \\
= & \frac{1}{K}\sum_{k=1}^K \mathcal{G}^{R-1}_k(\theta^{*}) - \theta^{*} - \frac{\eta}{K} \sum_{k=1}^K \nabla f_k (\mathcal{G}^{R-1}_k(\theta^{*})) \\
& \hspace{3em} \vdots \\
= & \frac{1}{K}\sum_{k=1}^K \mathcal{G}^{0}_k(\theta^{*}) - \theta^{*} - \frac{\eta}{K} \sum\limits_{i=1}^R \sum\limits_{k=1}^K \nabla f_k( \mathcal{G}_k^{i-1}(\theta^*) ) \\
= & - \frac{\eta}{K} \sum\limits_{i=1}^R \sum\limits_{k=1}^K \nabla f_k( \mathcal{G}_k^{i-1}(\theta^*) ).
\end{align*}
这样,我们就证明了$\theta^*$满足不动点条件~\eqref{eq:fedgd-fixed-pt}.
\item[(2)] 待写。。。
\end{itemize}
\end{proof}

待写。。。

\begin{algorithm}[!htb]
% \SetAlgoNoLine
% \DontPrintSemicolon
\SetKwInOut{Input}{Input}
\Input{{\bfseries proximal solvers} $\texttt{prox\_update}_k: \R^d \to \R^d$}
{\bfseries Initiation:} parameters $\theta^{(0)} \in \R^d$,\;
\For{each round $t = 0, 1, \cdots$}{
    broadcast $\theta^{(t)}$ to clients $k \in [K]$\;
    \For{each client $k = 1, \cdots, K$ {\bfseries in parallel}}{
        $\theta_k^{(t)} \gets \theta^{(t)}$\;
        $\theta_k^{(t+1/2)} \gets$ $\texttt{prox\_update}_k(2\theta^{(t)} - \theta_k^{(t)})$ \tcc*[h]{local prox step}\;
        $\theta_k^{(t+1)} \gets$ $\theta_k^{(t)} + 2(\theta_k^{(t+1/2)} - \theta^{(t)})$ \tcc*[h]{local centering step}\;
        send $\theta_k^{(t+1)}$ to server\;
        }
    {\bfseries Server Update:}\;
    \Indp
    $\theta^{(t+1)} \gets \frac{1}{K} \sum\limits_{k=1}^K \theta_k^{(t+1)}$ \tcc*[h]{compute global average}\;
    \If{meet convergent criteria}{
        $\theta^* \gets \theta^{(t+1)}$\;
        {\bfseries break}\;
    }
}
% return $\theta^*$\;
\caption{联邦分裂算法\texttt{FedSplit}\cite{pathak2020fedsplit}的伪代码}
\label{algo:fedsplit}
\end{algorithm}



我们从最优化理论的角度对FedSplit算法进行分析。事实上,我们考虑的优化问题~\eqref{eq:fl-basic}~可以等价地改写为如下的共识优化问题(注意与式~\eqref{eq:fl-basic-constraint}~的细微差别)
\begin{equation}
\label{eq:fedsplit-constraint}
\begin{array}{cl}
\minimize & F(\Theta) := \sum\limits_{k=1}^K w_k f_k(\theta_k), \\
\text{subject to} & A \Theta = 0,
\end{array}
\end{equation}
其中
\begin{equation*}
A = \begin{pmatrix} I_d & -I_d & & & \\ & I_d & -I_d & & \\ & & \ddots & \ddots & \\ & & & \ddots & -I_d \\ -I_d & & & & I_d \end{pmatrix}, ~ \Theta = \begin{pmatrix} \theta_1 \\ \vdots \\ \theta_K \end{pmatrix}, ~ \theta_1, \ldots, \theta_K \in \R^d.
\end{equation*}
约束优化问题~\eqref{eq:fedsplit-constraint}~的拉格朗日函数为
\begin{equation}
\label{eq:fedsplit-lagrangian}
\mathcal{L}(\Theta, \Lambda) = F(\Theta) - \langle \Lambda, A\Theta \rangle = \sum\limits_{k=1}^K w_k f_k(\theta_k) - \langle \Lambda, A\Theta \rangle,
\end{equation}
其中$\Lambda = \begin{pmatrix} \lambda_1 \\ \vdots \\ \lambda_K \end{pmatrix} \in \R^{Kd}$为对偶变量。考虑上述问题的一阶最优性条件$\nabla F(\Theta) - A^T \Lambda,$ 即
\begin{equation*}
\nabla F(\Theta) - \begin{pmatrix} \lambda_1 - \lambda_K \\ \lambda_2 - \lambda_1 \\ \vdots \\ \lambda_K - \lambda_{K-1} \end{pmatrix} = 0.
\end{equation*}
上式是一个所谓的单调包含问题(Monotone Inclusion Problem)
\begin{equation}
\label{eq:fedsplit-mono-incl}
0 \in \nabla F(\Theta) + \mathcal{N}_{\mathcal{E}}(\Theta)
\end{equation}
其中
\begin{equation*}
\mathcal{N}_{\mathcal{E}}(\Theta) = \begin{cases} \mathcal{E}^{\perp} & \text{ if } \Theta \in \mathcal{E} \\ \emptyset & \text{ otherwise } \end{cases}
\end{equation*}
是凸集$\mathcal{E} = \{ \Theta \in \R^{Kd} ~ | ~ \theta_1 = \cdots = \theta_K\}$在点$\Theta$处的法锥(Normal Cone)。

我们回忆一下法锥(Normal Cone)的定义:
\begin{definition}[法锥(Normal Cone)]
设$\Omega \in \R^n$是$n$维欧氏空间中的非空凸集,$x \in \Omega$是其中一点,那么非空凸集$\Omega$在点$x$处的法锥(Normal Cone)为
\begin{equation}
\label{eq:def-normal-cone}
\mathcal{N}_{\Omega}(x) = \left\{ v ~ \middle| ~ \langle v, w - x \rangle \leqslant 0 ~ \forall w \in \Omega \right\}.
\end{equation}
\end{definition}


\begin{algorithm}[!htb]
% \SetAlgoNoLine
% \DontPrintSemicolon
\SetKwInOut{Input}{Input}
\Input{step size $s = \frac{1}{\mu} > 0$, $\alpha \in (0, 2)$, error bounds $\varepsilon_{k,0} \geqslant 0$}
{\bfseries Initiation:}\;
\Indp
    {\bfseries Init server:} global model parameters $\theta^{(0)} \in \operatorname{dom}(f)$, $\overline{\theta}^{(0)} = \widetilde{\theta}^{(0)} = \omega^{(0)} = \theta^{(0)} \in \R^d$\;
    {\bfseries Init clients:} $\omega_k^{(0)} = \theta^{(0)}$, $\theta_k^{(0)} \approx \prox_{f_k, \mu}(\omega_k^{(0)})$, $\widehat{\theta}_k^{(0)} = 2\theta_k^{(0)} - \omega_k^{(0)}, ~ \forall k \in [K]$\;
\Indm
\For{$t = 0, 1, \cdots, T-1$}{
    $\mathcal{S}^{(t)} \gets$ (random set of clients) $\subseteq [K]$\;
    each client $k \in \mathcal{S}^{(t)}$ receives $\overline{\theta}^{(t)}$ from server \tcc*[h]{communication}\;
    \For{each client $k \in \mathcal{S}^{(t)}$ {\bfseries in parallel}}{
        choose $\varepsilon_{k,t+1} \geqslant 0$\;
        update $\omega_k^{(t+1)} \gets \omega_k^{(t)} + \alpha(\overline{\theta}^{(t)} - \theta_k^{(t)}),$ \;
        $\theta_k^{(t+1)} \approx \prox_{f_k, \mu}(y_k^{(t+1)})$  \tcc*[h]{inexact local prox step with error bound $\varepsilon_{k,0}$}\;
        $\widehat{\theta}_k^{(t+1)} \gets 2\theta_k^{(t+1)} - \omega_k^{(t+1)}$ \;
        send $\Delta \widehat{\theta}_k^{(t)} = \widehat{\theta}_k^{(t+1)} - \widehat{\theta}_k^{(t)}$ to server\;
        }
    {\bfseries Server Update:}\;
    \Indp
        $\omega^{(t+1)} \gets \omega^{(t)} + \alpha (\overline{\theta}^{(t)} - \omega^{(t)})$\;
        $\widetilde{\theta}^{(t+1)} \gets \widetilde{\theta}^{(t)} + \frac{1}{K}\sum_{k \in \mathcal{S}^{(t)}} \Delta \widehat{\theta}_k^{(t)}$\;
        $\overline{\theta}^{(t+1)} \gets \prox_{g, \frac{K+1}{Ks}} \left( \frac{K}{K+1} \widetilde{\theta}^{(t+1)} + \frac{1}{K+1} \omega^{(t+1)} \right)$ \;
    \Indm
}
\caption{联邦DR分裂算法\texttt{FedDR}\cite{tran2021feddr}伪代码}
\label{algo:feddr}
\end{algorithm}

