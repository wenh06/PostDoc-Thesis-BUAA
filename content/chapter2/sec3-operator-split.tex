\section{联邦学习中的算子分裂算法}
\addcontentsline{toe}{section}{{2.3\ \ Operator Splitting Algorithms in Federated Learning}\numberline\,}
\label{sec:chap2-operator-split}

待写。。。。

\begin{algorithm}[ht]
% \SetAlgoNoLine
% \DontPrintSemicolon
\SetKwInOut{Input}{Input}
\Input{{\bfseries proximal solvers} $\texttt{prox\_update}_j: \R^d \to \R^d$}
{\bfseries Initiation:} parameters $\theta^{(0)} \in \R^d$,\;
\For{each round $t = 0, 1, \cdots$}{
    \For{each client $k = 1, \cdots, K$ {\bfseries in parallel}}{
        $\theta_k^{(t)} \gets \theta^{(t)}$\;
        $\theta_k^{(t+1/2)} \gets$ $\texttt{prox\_update}_k(2\theta^{(t)} - \theta_k^{(t)})$ \tcc*[h]{local prox step}\;
        $\theta_k^{(t+1)} \gets$ $\theta_k^{(t)} + 2(\theta_k^{(t+1/2)} - \theta^{(t)})$ \tcc*[h]{local centering step}
        }
    {\bfseries Server Update:}\;
    \Indp
    $\theta^{(t+1)} \gets \overline{\theta}^{(t+1)}$ \tcc*[h]{compute global average}\;
    \If{meet convergent criteria}{
        $\theta^* \gets \theta^{(t+1)}$\;
        {\bfseries break}\;
    }
}
% return $\theta^*$\;
\caption{\texttt{FedSplit}\cite{pathak2020fedsplit}算法的伪代码}
\label{algo:fedsplit}
\end{algorithm}



我们从最优化理论的角度对FedSplit算法进行分析。事实上,我们考虑的优化问题\eqref{eq:fl-basic}可以等价地改写为如下的共识优化问题(注意与式\eqref{eq:fl-basic-constraint}的细微差别)
\begin{equation}
\label{eq:fedsplit-constraint}
\begin{array}{cl}
\minimize & F(\Theta) := \sum\limits_{k=1}^K w_k f_k(\theta_k), \\
\text{subject to} & A \Theta = 0,
\end{array}
\end{equation}
其中
\begin{equation*}
A = \begin{pmatrix} I_d & -I_d & & & \\ & I_d & -I_d & & \\ & & \ddots & \ddots & \\ & & & \ddots & -I_d \\ -I_d & & & & I_d \end{pmatrix}, ~ \Theta = \begin{pmatrix} \theta_1 \\ \vdots \\ \theta_K \end{pmatrix}, ~ \theta_1, \ldots, \theta_K \in \R^d.
\end{equation*}
约束优化问题\eqref{eq:fedsplit-constraint}的拉格朗日函数为
\begin{equation}
\label{eq:fedsplit-lagrangian}
\mathcal{L}(\Theta, \Lambda) = F(\Theta) - \langle \Lambda, A\Theta \rangle = \sum\limits_{k=1}^K w_k f_k(\theta_k) - \langle \Lambda, A\Theta \rangle,
\end{equation}
其中$\Lambda = \begin{pmatrix} \lambda_1 \\ \vdots \\ \lambda_K \end{pmatrix} \in \R^{Kd}$为对偶变量。考虑上述问题的一阶最优性条件$\nabla F(\Theta) - A^T \Lambda,$ 即
\begin{equation*}
\nabla F(\Theta) - \begin{pmatrix} \lambda_1 - \lambda_K \\ \lambda_2 - \lambda_1 \\ \vdots \\ \lambda_K - \lambda_{K-1} \end{pmatrix} = 0.
\end{equation*}
上式是一个所谓的单调包含问题(Monotone Inclusion Problem)
\begin{equation}
\label{eq:fedsplit-mono-incl}
0 \in \nabla F(\Theta) + \mathcal{N}_{\mathcal{E}}(\Theta)
\end{equation}
其中
\begin{equation*}
\mathcal{N}_{\mathcal{E}}(\Theta) = \begin{cases} \mathcal{E}^{\perp} & \text{ if } \Theta \in \mathcal{E} \\ \emptyset & \text{ otherwise } \end{cases}
\end{equation*}
是凸集$\mathcal{E} = \{ \Theta \in \R^{Kd} ~ | ~ \theta_1 = \cdots = \theta_K\}$在点$\Theta$处的法锥(Normal Cone)。

我们回忆一下法锥(Normal Cone)的定义:
\begin{definition}[法锥(Normal Cone)]
设$\Omega \in \R^n$是$n$维欧氏空间中的非空凸集,$x \in \Omega$是其中一点,那么非空凸集$\Omega$在点$x$处的法锥(Normal Cone)为
\begin{equation}
\label{eq:def-normal-cone}
\mathcal{N}_{\Omega}(x) = \left\{ v ~ \middle| ~ \langle v, w - x \rangle \leqslant 0 ~ \forall w \in \Omega \right\}.
\end{equation}
\end{definition}


\begin{algorithm}[ht]
% \SetAlgoNoLine
% \DontPrintSemicolon
{\bfseries Initiation:} $x^{(0)} \in \operatorname{dom}(f)$, $s > 0$, $\alpha > 0$, $\varepsilon_{i,0} \geqslant 0$ \\
\hspace{2em}Init server: $\overline{x}^{(0)} = \widetilde{x}^{(0)} = y^{(0)} = x^{(0)}$\\
\hspace{2em}Init clients: $y_i^{(0)} = x^{(0)}$, {\color{red}$x_i^{(0)} \approx \operatorname{prox}_{sf_i}(y_i^{(0)})$}, $\widehat{x}_i^{(0)} = 2x_i^{(0)} - y_i^{(0)}$\;
\For{$t = 0, 2, \cdots, T-1$}{
    sample $\mathcal{S}^{(t)} \subseteq [K]$ \tcc*[h]{active clients}\;
    each user $k \in \mathcal{S}^{(t)}$ receives $\overline{x}^{(t)}$ from server \tcc*[h]{communication}\;
    \For{each user $i\in\mathcal{S}^{(t)}$ {\bfseries in parallel}}{
        choose $\varepsilon_{k,t+1} \geqslant 0$, update 
        $y_k^{(t+1)} \gets y_k^{(t)} + \alpha(\overline{x}^{(t)} - x_k^{(t)}),$ \;
        $x_k^{(t+1)} \approx \operatorname{prox}_{sf_k}(y_k^{(t+1)}), \ \widehat{x}_k^{(t+1)} \gets 2x_k^{(t+1)} - y_k^{(t+1)}$ \;
        send $\Delta \widehat{x}_k^{(t)} = \widehat{x}_k^{(t+1)} - \widehat{x}_k^{(t)}$ to server\;
        }
    {\bfseries Server Update:}\;
    \Indp
        $y^{(t+1)} \gets y^{(t)} + \alpha (\overline{x}^{(t)} - y^{(t)})$\;
        $\widetilde{x}^{(t+1)} \gets \widetilde{x}^{(t)} + \frac{1}{n}\sum_{k \in \mathcal{S}^{(t)}} \Delta \widehat{x}_k^{(t)}$\;
        $\overline{x}^{(t+1)} \gets \operatorname{prox}_{\frac{Ks}{K+1}g} (\frac{K}{K+1}\widetilde{x}^{(t+1)} + \frac{1}{K+1}y^{(t+1)})$ \;
}
% return $\overline{x}^{(T+1)}$\;
\caption{FedDR\cite{tran2021feddr}算法伪代码}
\end{algorithm}

