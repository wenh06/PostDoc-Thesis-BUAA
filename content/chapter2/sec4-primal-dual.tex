\section{联邦学习中的原始对偶算法}
\addcontentsline{toe}{section}{{2.4\ \ Primal-Dual Algorithms in Federated Learning}\numberline\,}
\label{sec:chap2-primal-dual}

% NOT finished

在传统的最优化方法中,原始-对偶算法(Primal-Dual Algorithms)也是一类常用的算法。我们考虑格式为~\eqref{eq:fl-basic-constraint}~的带等式线性约束的共识优化问题(注意与~\eqref{eq:fl-basic-constraint}~的细微差别)
\begin{equation}
\label{eq:fedpd-constraint}
\begin{array}{cl}
\minimize & F(\Theta) := \sum\limits_{k=1}^K f_k(\theta_k), \\
\text{subject to} & \theta_k = \theta, ~ k = 1, \ldots, K.
\end{array}
\end{equation}
其中$\Theta = \col(\theta_1, \cdots, \theta_K), ~ \theta, \theta_1, \ldots, \theta_K \in \R^d.$ 约束优化问题~\eqref{eq:fedpd-constraint}~的增广拉格朗日函数(Augmented Lagrangian, AL)为
\begin{equation}
\label{eq:al}
\mathcal{L}(\Theta, \Lambda) = F(\Theta) - \sum\limits_{k=1}^K \left\{ \langle \lambda_k, \theta_k \rangle + \frac{1}{2\eta} \lVert \theta_k - \theta \rVert^2 \right\} = \sum\limits_{k=1}^K \mathcal{L}_k(\theta, \theta_k, \lambda_k)
\end{equation}
其中
\begin{equation}
\label{eq:al-local}
\mathcal{L}_k(\theta, \theta_k, \lambda_k) = f_k(\theta_k) + \langle \lambda_k, \theta_k - \theta \rangle + \frac{1}{2\eta} \lVert \theta_k - \theta \rVert^2,
\end{equation}
$\Lambda = \col(\lambda_1, \ldots, \lambda_K)$被称作对偶变量(Dual Variable)或者拉格朗日乘子(Lagrangian Multiplier)。

待写

文献\parencite{zhang2020fedpd}研究了

\begin{algorithm}[!htb]
% \SetAlgoNoLine
% \DontPrintSemicolon
\SetKwInOut{Input}{Input}
\Input{step size $s = \frac{1}{\mu} > 0,$ skip probability $p \in [0, 1)$}
{\bfseries Initiation:}\;
\Indp
    {\bfseries Init server:} global parameters $\theta^{(0)} \in \R^d$\;
    {\bfseries Init clients:} local parameters$\theta_{k,0}^{(0)} \in \R^d,$ dual variables $\lambda_k^{(0)} \in \R^d,$ $\forall k \in [K]$\;
\Indm
\For{each round $t = 0, 1, \cdots$}{
    \For{each client $k = 1, \cdots, K$ {\bfseries in parallel}}{
        $\theta_k^{(t+1)} \gets \operatorname{\mathbf{Oracle}}_k(\mathcal{L}_k(\theta_{k,0}^{(t)}, \theta_k, \lambda_k^{(t)}), \theta_k^{(t)})$  \tcc*[h]{$\operatorname{\mathbf{Oracle}}_k$ can be SGD, etc.}\;
        $\lambda_k^{(t+1)} \gets \lambda_k^{(t)} + \frac{1}{s} (\theta_k^{(t+1)} - \theta_{k,0}^{(t)})$ \tcc*[h]{dual update step}\;
        $\theta_{k,0}^{(t+\frac{1}{2})} \gets \theta_k^{(t+1)} + s \lambda_k^{(t+1)}$
        }
    with probability $1 - p$ do global communication\;
    \Indp
    client $k$ send $\theta_{k,0}^{(t+\frac{1}{2})}$ to server $\forall k \in [K]$\;
    {\bfseries Server Update:} $\theta^{(t+1)} \gets \frac{1}{K} \sum\limits_{k=1}^K \theta_{k,0}^{(t+\frac{1}{2})}$ \tcc*[h]{compute global average}\;
    Server broadcast $\theta^{(t+1)}$ to clients $k \in [K]$\;
    On client $k:$ $\theta^{(t+1)}_{k,0} \gets \theta^{(t+1)}, ~ \forall k \in [K]$\;
    \Indm
    with probability $p$ skip global communication:\;
    \Indp
    {\bfseries Client Update:} $\theta^{(t+1)}_{k,0} \gets \theta_{k,0}^{(t+\frac{1}{2})}, ~ \forall k \in [K]$\;
    \tcc{On server, $\theta^{(t+1)} \gets \theta^{(t)}$}
    \Indm
}
\caption{联邦原始对偶算法\texttt{FedPD}\cite{zhang2020fedpd}的伪代码}
\label{algo:fedpd}
\end{algorithm}


待写。。。。
