\section{联邦学习中的跳步算法}
\addcontentsline{toe}{section}{{2.5\ \ Skipping Algorithms in Federated Learning}\numberline\,}
\label{sec:chap2-skip-alg}

% NOT finished

我们已经在\S\ref{sec:chap2-primal-dual}~已经接触到了使用``跳步''(Skipping)更新技术的联邦学习算法了。事实上,最早的联邦学习算法,联邦平均算法\texttt{FedAvg}\cite{mcmahan2017fed_avg}在子节点上的内循环执行多步梯度下降的做法,在某种意义上也可以视作是跳步更新的方法。文献\cite{proxskip,proxskip-vr}进一步深入探讨了跳步更新这一技术在联邦学习优化算法中的应用。

再一次考虑建模为共识优化~\eqref{eq:fl-basic-constraint}~的联邦学习问题
\begin{equation*}
\begin{array}{rl}
\minimize & F(\Theta) = \frac{1}{K} \sum\limits_{k=1}^K f_k(\theta_k), \\
\text{subject to} & \Theta \in \mathcal{E}, \\
\text{where} & \Theta = \col(\theta_1, \cdots, \theta_K) \in \R^{Kd}, \\
& \mathcal{E} = \{ \Theta ~|~ \theta_1 = \cdots = \theta_K \}.
\end{array}
\end{equation*}
我们已经在\S\ref{subsec:chap2-overview-fedavg-opt}~对联邦平均算法\texttt{FedAvg}进行理论分析,以及在\S\ref{sec:chap2-operator-split}~讨论联邦学习算子分裂算法等多个地方使用了这一优化模型。

\begin{algorithm}[ht]
% \SetAlgoNoLine
% \DontPrintSemicolon
\SetKwInOut{Input}{Input}
\Input{xx}
{\bfseries Initiation:}\;
\caption{算法\texttt{ProxSkip}\cite{proxskip}的伪代码}
\label{algo:proxskip}
\end{algorithm}

