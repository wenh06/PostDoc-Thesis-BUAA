\section{联邦学习中的优化算法}
\addcontentsline{toe}{section}{{2.1\ \ Overview of Optimization Algorithms in Federated Learning}\numberline\,}
\label{sec:chap2-overview}

优化算法自从联邦学习这一概念诞生起,便一直是相关研究的中心问题之一。联邦学习开创性的文献\cite{mcmahan2017fed_avg}中,最重要也是最为人所熟知的便是优化算法\texttt{FederatedAveraging}(简记为\texttt{FedAvg})的提出。这种算法的核心思想,是充分利用子节点的计算能力,在执行完一定轮数的随机梯度下降(Stochastic Gradient Descent, SGD)之后,由中心节点收集各子节点发送过来的模型参数,进行平均(Averaging),之后再将平均之后的模型参数广播(Broadcast)给子节点进行下一轮迭代。这样,规避了每一轮SGD都进行通信带来的巨大的通信开销,在加速模型训练的同时,数值上在某些情况下还有更好的收敛性。这实际上是一种比较朴素与简单的``跳步''的思想,这种思想也在随后的研究\cite{zhang2020fedpd, proxskip, proxskip-vr}也得到了进一步的发展。

联邦学习优化算法最重要的设计原则,是计算效率以及通信效率并重,甚至很多时候通信效率是更重要的一个方面。这也是联邦学习与传统的分布式优化(Distributed Optimization)最显著的区别之一。实际上,分布式优化在联邦学习被提出之前就是一个被研究得比较多的问题,从具体的问题,例如分布式主成分分析(Principal Component Analysis, PCA)\cite{dist_pca_2014_nips},到一般的算法理论\cite{boyd2011distributed}都有研究人员进行了研究。当时这些研究往往仅从计算效率以及效果出发,往往不考虑通信效率、通信保密性等问题。文献\cite{mcmahan2017fed_avg}正是从实际问题出发,发现了这些需求,基于一种分布式梯度下降算法\cite{chen2016_revisit}(该算法在文献\cite{mcmahan2017fed_avg}中被称作\texttt{FedSGD}算法),做了上文提到的改进而提出了\texttt{FedAvg}算法。

待写。。。。

沿用式\eqref{eq:general-hfl}中的记号,本文考虑联邦学习中的优化问题,其最基本的格式如下
\begin{equation}
\label{eq:fl-basic-dist}
\begin{array}{cl}
\minimize\limits_{\theta \in \R^d} & f(\theta) = \expectation\limits_{k \sim {\mathcal{P}}} [f_k(\theta)], \\
\text{where} & f_k(\theta) = \expectation\limits_{(x, y) \sim \mathcal{D}_k} [\ell_k(\theta; x, y)],
\end{array}
\end{equation}
假设我们令$\mathcal{P} = \{1, 2, \ldots, K\},$ 则上述模型可以简记为
\begin{equation}
\label{eq:fl-basic}
\begin{array}{cl}
\minimize\limits_{\theta \in \R^d} & f(\theta) = \sum\limits_{k=1}^K w_k f_k(\theta).
\end{array}
\end{equation}
对于$f_k,$ 我们一般都假设它满足如下几条性质
\begin{itemize}
    \item 待写
    \item 待写
    \item 待写
\end{itemize}

更方便与自然的做法,是将问题\eqref{eq:fl-basic}写成约束优化问题的格式:
\begin{equation}
\label{eq:fl-basic-constraint}
\begin{array}{cl}
xxx
\end{array}
\end{equation}
