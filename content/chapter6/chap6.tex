%\mbox{}\newpage
\chapter{\hspace{-1mm}\bf 数值实验}
\label{chap6}
\addcontentsline{toe}{chapter}{{{\bf Chapter \currentchapter \ 
Numerical Experiments }}\numberline\,}
\markboth{第 \currentchapter \ 章
数值实验}{北京航空航天大学博士后研究工作报告}

%%%%%%%%%%%%%%%%%%%%%%%%%%%%%%%%%%%%%%%%%%%%%%%%%%%%%%%%%%%

本章对本文前面几个章节介绍的部分联邦学习算法进行数值上的评测。评测主要从算法的整体效果 (在测试集上的准确度、损失),对于子节点掉队比例的鲁棒性以及对学习率等超参数的敏感性等几个方面进行。为了排除随机因素的干扰,每一组实验都使用了5个不同的随机数种子 (一般都被设为$0, 1, 2, 3, 4$) 进行重复试验。由于本文成文时间的关系,本章的数值试验大部分使用的是上一章\S\ref{sec:chap5-datasets}中介绍的小规模的联邦数据集\texttt{FedProxFEMNIST}。选择这个数据集是基于如下的原因:待写。。。。


\section{联邦学习算法整体效果的数值评测}
\addcontentsline{toe}{section}{{\currentchapter .1\ \ Numerical Evaluation of the Overall Effectiveness of Federated Learning Algorithms}\numberline\,}
\label{sec:chap6-overall}

% NOT finished
% NOT indexed

待写。。。。

\section{待写}
\addcontentsline{toe}{section}{{\currentchapter .2\ \ To write}\numberline\,}

% NOT finished
% NOT indexed

待写。。。。

\section{待写}
\addcontentsline{toe}{section}{{\currentchapter .3\ \ To write}\numberline\,}

% NOT finished
% NOT indexed

待写。。。。
