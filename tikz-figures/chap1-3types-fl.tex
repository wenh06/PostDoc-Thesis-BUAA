\begin{figure}[H]
\centering
\begin{boxedminipage}{0.95\textwidth}
\centering
\begin{minipage}{0.29\textwidth}
\begin{tikzpicture}
\draw (0, 0) rectangle (3.8, 4.5);
\node at (1.95, 5.5) {横向联邦学习};
\draw[fill=pink] (0, 0) rectangle (3.5, 1.1) node[pos=.5] {client $i$};
\draw[fill=green] (0.2, 3.3) rectangle (3.8, 4.5) node[pos=.5] {client $j$};
\draw[fill=yellow] (0.3, 0.9) rectangle (3.7, 3.6) node[pos=.4] {client $k$};
\node at (1.95, -0.4) {样本特征空间};
\end{tikzpicture}
\end{minipage}
\begin{minipage}{0.29\textwidth}
\begin{tikzpicture}
\draw (0,0) rectangle (3.8, 4.5);
\node at (1.95, 5.5) {纵向联邦学习};
\draw[fill=pink] (0, 0) rectangle (1.4, 4.2) node[pos=.5, rotate=-90] {client $i$};
\draw[fill=green] (1.95, 0) rectangle (3.8, 4.5) node[pos=.5, rotate=-90] {client $j$};
\draw[fill=yellow] (1.3, 0.2) rectangle (2.1, 4.4) node[pos=.4, rotate=-90] {client $k$};
\node at (1.95, -0.4) {样本特征空间};
\end{tikzpicture}
\end{minipage}
\begin{minipage}{0.29\textwidth}
\begin{tikzpicture}
\draw (0,0) rectangle (3.8, 4.5);
\node at (1.95, 5.5) {联邦迁移学习};
\draw[fill=yellow] (0.9, 1.2) rectangle (2.2, 4.1) node[pos=.5] {client $k$};
\draw[fill=pink] (0, 0) rectangle (1.3, 1.6) node[pos=.5] {client $i$};
\draw[fill=green] (1.9, 3.6) rectangle (3.8, 4.5) node[pos=.5] {client $j$};
\node at (1.95, -0.4) {样本特征空间};
\end{tikzpicture}
\end{minipage}
\begin{minipage}{0.05\textwidth}
\begin{tikzpicture}
% \coordinate (sample_node) at (0,0);
% \node[text width=1, left=0.5cm of sample_node.west] at () {样本维度};
\node[rotate=-90] () {样本编号空间};
\end{tikzpicture}
\end{minipage}
\end{boxedminipage}
\caption{三类联邦学习的样本特征空间与样本编号空间分布的示意图}
\label{fig:three-types-fl}
\end{figure}
