\begin{figure}[H]
\centering
\begin{tikzpicture}[scale=1.8]
% \coordinate (origin) at (0, 0);
\coordinate (rect1) at (-4, -2.1);
\coordinate (rect2) at (3.6, 2.3);
\draw (rect1) rectangle (rect2);
\node at (-3.2, -1.9) {$\dom f_k$};
\begin{scope}
\clip (rect1) rectangle (rect2);
\draw[] plot [smooth cycle] coordinates {(-1, 0) (-0.7, -0.1) (-0.3, 0.2) (-0.5, 0.3) (-1.1, 0.1)};
% \draw[] plot [smooth cycle] coordinates {(-1.3, -0.3) (-0.6, -0.6) (0.3, 0.7) (-0.1, 0.9) (-0.7, 0.7) (-1.6, 0.2)};
\draw[] plot [smooth cycle] coordinates {(-1.6, -0.5) (-0.2, -1.2) (0.9, 1.2) (0.6, 1.6) (-0.7, 1.2) (-2.2, 0.3)};
\draw[] plot [smooth cycle] coordinates {(-2.4, -1.1) (0.6, -2.6) (1.9, 1.9) (1.1, 2.7) (-0.9, 1.9) (-3.1, 0.1)};
\end{scope}
\node at (0.9, -0.5) (global) [circle, fill=black, inner sep=0pt ,minimum size=5pt, label=below:{$\theta^{(t)}$}] {};
\node at (1.2, 0.5) (local) [circle, fill=black, inner sep=0pt, minimum size=5pt, label=above:{$\theta_k^{(t)}$}] {};
\coordinate (min1) at (-0.7, 0.1);
\draw plot[only marks, mark=triangle*, mark size=4pt, thick] coordinates {(min1)};
\coordinate (min2) at (0.1, -0.6);
\draw plot[only marks, mark=star, mark size=4pt, thick] coordinates {(min2)};
\path (local) edge [draw, dashed, -{Stealth}] ($(local)!0.6!(min1)$);
\node at ($(local)!0.6!(min1)$) (grad) [label=above:{$\mathfrak{g}_k^{(t)}$}] {};
\path (local) edge [draw, dashed, -{Stealth}] ($(local)!0.7!(min2)$);
\node at (0.3, 0.0) (next) [circle, fill=black, inner sep=0pt, minimum size=5pt, label=left:{$\theta_k^{(t+1)}$}] {};
\path (local) edge [draw, thick, -{Stealth}, decorate, decoration={snake, amplitude=1.5pt, pre length=4pt, post length=3pt}] (next);
\begin{scope}
\clip (rect1) rectangle (rect2);
\draw[dashed, thin] (global) circle (0.7);
% \draw[dashed, thin] (global) circle (1.1);
\draw[dashed, thin] (global) circle (1.9);
\draw[dashed, thin] (global) circle (3.2);
\end{scope}
\end{tikzpicture}
\caption{动态正则化联邦学习算法\texttt{FedDyn}子节点模型参数更新示意图}
\label{fig:feddyn}
\end{figure}
