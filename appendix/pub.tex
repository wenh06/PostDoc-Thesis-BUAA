\chapter*{博士后在站期间的主要成果}
\addcontentsline{toc}{chapter}{{博士后在站期间的主要成果}\numberline~~~}
\addcontentsline{toe}{chapter}{{Main achievements during the postdoctoral program}\numberline ~~~}
\markboth{博士后在站期间的主要成果}{博士后在站期间的主要成果}\headheight=15.24pt%5mm

\noindent{\larger \heiti 基金项目:}

\begin{itemize}
    \item 面向脊柱穿刺消融手术的多模态图像导航关键算法研究,数学天元基金项目/数学天元基金/数学与医疗健康交叉重点专项,参与,起止时间:2022/01 -- 2023/12。
    \item 数字电路物理设计自动化中关键数学问题研究,国家重点研发计划,参与,起止时间:2022/05 -- 2025/04。
\end{itemize}

\noindent{\larger \heiti 科研论文:}

\begin{enumerate}
    \item WEN H, KANG J. Hybrid Arrhythmia Detection on Varying-Dimensional Electrocardiography: Combining Deep Neural Networks and Clinical Rules[C]//2021 Computing in Cardiology (CinC): vol. 48. Institute of Electrical and Electronics Engineers (IEEE), 2021. DOI: 10.23919/cinc53138.2021.9662801.
    \item KANG J, WEN H. A Study on Several Critical Problems on Arrhythmia Detection using Varying-Dimensional Electrocardiography[J]. Physiological Measurement, 2022, 43(6): 064007. DOI: 10.1088/1361-6579/ac6aa3.
    \item WEN H, KANG J. A Novel Deep Learning Package for Electrocardiography Research[J]. Physiological Measurement, 2022, 43(11): 115006. DOI: 10.1088/1361-6579/ac9451.
    \item WEN H, KANG J. A Comparative Study on Neural Networks for Paroxysmal Atrial Fibrillation Events Detection from Electrocardiography[J]. Journal of Electrocardiology, 2022, 75: 19-27. DOI: 10.1016/j.jelectrocard.2022.10.002.
    \item GU E, CHEN Y, WEN H, CAI X, HAN D. Novel Clustered Federated Learning Based on Local Loss. 2023. Submitted.
\end{enumerate}

% \patchcmd{\thebibliography}{\section*}{\chapter*}{}{}

% \begin{refsection}[publications.bib]
% \nocite{wen_cinc2021, Kang_2022_cinc2021_iop, torch_ecg_paper, Wen_cpsc2021, wen_cinc2022}
% \printbibliography[title={\heiti 科研论文}]
% \end{refsection}

